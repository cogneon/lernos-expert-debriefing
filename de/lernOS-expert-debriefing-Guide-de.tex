%%
% Copyright (c) 2017 - 2025, Pascal Wagler;
% Copyright (c) 2014 - 2025, John MacFarlane
%
% All rights reserved.
%
% Redistribution and use in source and binary forms, with or without
% modification, are permitted provided that the following conditions
% are met:
%
% - Redistributions of source code must retain the above copyright
% notice, this list of conditions and the following disclaimer.
%
% - Redistributions in binary form must reproduce the above copyright
% notice, this list of conditions and the following disclaimer in the
% documentation and/or other materials provided with the distribution.
%
% - Neither the name of John MacFarlane nor the names of other
% contributors may be used to endorse or promote products derived
% from this software without specific prior written permission.
%
% THIS SOFTWARE IS PROVIDED BY THE COPYRIGHT HOLDERS AND CONTRIBUTORS
% "AS IS" AND ANY EXPRESS OR IMPLIED WARRANTIES, INCLUDING, BUT NOT
% LIMITED TO, THE IMPLIED WARRANTIES OF MERCHANTABILITY AND FITNESS
% FOR A PARTICULAR PURPOSE ARE DISCLAIMED. IN NO EVENT SHALL THE
% COPYRIGHT OWNER OR CONTRIBUTORS BE LIABLE FOR ANY DIRECT, INDIRECT,
% INCIDENTAL, SPECIAL, EXEMPLARY, OR CONSEQUENTIAL DAMAGES (INCLUDING,
% BUT NOT LIMITED TO, PROCUREMENT OF SUBSTITUTE GOODS OR SERVICES;
% LOSS OF USE, DATA, OR PROFITS; OR BUSINESS INTERRUPTION) HOWEVER
% CAUSED AND ON ANY THEORY OF LIABILITY, WHETHER IN CONTRACT, STRICT
% LIABILITY, OR TORT (INCLUDING NEGLIGENCE OR OTHERWISE) ARISING IN
% ANY WAY OUT OF THE USE OF THIS SOFTWARE, EVEN IF ADVISED OF THE
% POSSIBILITY OF SUCH DAMAGE.
%%

%%
% This is the Eisvogel pandoc LaTeX template.
%
% For usage information and examples visit the official GitHub page:
% https://github.com/Wandmalfarbe/pandoc-latex-template
%%
\newcounter{none}
% Options for packages loaded elsewhere
\PassOptionsToPackage{unicode}{hyperref}
\PassOptionsToPackage{hyphens}{url}
\PassOptionsToPackage{dvipsnames,svgnames,x11names,table}{xcolor}
\documentclass[
  ngerman,
  paper=a4,
  ,captions=tableheading
]{scrartcl}
\usepackage{xcolor}
\usepackage[margin=2.5cm,includehead=true,includefoot=true,centering,]{geometry}
\usepackage{amsmath,amssymb}


% add backlinks to footnote references, cf. https://tex.stackexchange.com/questions/302266/make-footnote-clickable-both-ways
\usepackage{footnotebackref}
\setcounter{secnumdepth}{5}
\usepackage{iftex}
\ifPDFTeX
  \usepackage[T1]{fontenc}
  \usepackage[utf8]{inputenc}
  \usepackage{textcomp} % provide euro and other symbols
\else % if luatex or xetex
  \usepackage{unicode-math} % this also loads fontspec
  \defaultfontfeatures{Scale=MatchLowercase}
  \defaultfontfeatures[\rmfamily]{Ligatures=TeX,Scale=1}
\fi
\usepackage{lmodern}
\ifPDFTeX\else
  % xetex/luatex font selection
\fi
% Use upquote if available, for straight quotes in verbatim environments
\IfFileExists{upquote.sty}{\usepackage{upquote}}{}
\IfFileExists{microtype.sty}{% use microtype if available
  \usepackage[]{microtype}
  \UseMicrotypeSet[protrusion]{basicmath} % disable protrusion for tt fonts
}{}

% Use setspace anyway because we change the default line spacing.
% The spacing is changed early to affect the titlepage and the TOC.
\usepackage{setspace}
\setstretch{1.2}
\makeatletter
\@ifundefined{KOMAClassName}{% if non-KOMA class
  \IfFileExists{parskip.sty}{%
    \usepackage{parskip}
  }{% else
    \setlength{\parindent}{0pt}
    \setlength{\parskip}{6pt plus 2pt minus 1pt}}
}{% if KOMA class
  \KOMAoptions{parskip=half}}
\makeatother
\usepackage{longtable,booktabs,array}
\usepackage{calc} % for calculating minipage widths
% Correct order of tables after \paragraph or \subparagraph
\usepackage{etoolbox}
\makeatletter
\patchcmd\longtable{\par}{\if@noskipsec\mbox{}\fi\par}{}{}
\makeatother
% Allow footnotes in longtable head/foot
\IfFileExists{footnotehyper.sty}{\usepackage{footnotehyper}}{\usepackage{footnote}}
\makesavenoteenv{longtable}
\usepackage{graphicx}
\makeatletter
\newsavebox\pandoc@box
\newcommand*\pandocbounded[1]{% scales image to fit in text height/width
  \sbox\pandoc@box{#1}%
  \Gscale@div\@tempa{\textheight}{\dimexpr\ht\pandoc@box+\dp\pandoc@box\relax}%
  \Gscale@div\@tempb{\linewidth}{\wd\pandoc@box}%
  \ifdim\@tempb\p@<\@tempa\p@\let\@tempa\@tempb\fi% select the smaller of both
  \ifdim\@tempa\p@<\p@\scalebox{\@tempa}{\usebox\pandoc@box}%
  \else\usebox{\pandoc@box}%
  \fi%
}
% Set default figure placement to htbp
% Make use of float-package and set default placement for figures to H.
% The option H means 'PUT IT HERE' (as  opposed to the standard h option which means 'You may put it here if you like').
\usepackage{float}
\floatplacement{figure}{H}
\makeatother
\ifLuaTeX
\usepackage[bidi=basic,shorthands=off,]{babel}
\else
\usepackage[bidi=default,shorthands=off,]{babel}
\fi
\ifLuaTeX
  \usepackage{selnolig} % disable illegal ligatures
\fi
\setlength{\emergencystretch}{3em} % prevent overfull lines
\providecommand{\tightlist}{%
  \setlength{\itemsep}{0pt}\setlength{\parskip}{0pt}}
\usepackage{bookmark}
\IfFileExists{xurl.sty}{\usepackage{xurl}}{} % add URL line breaks if available
\urlstyle{same}
% Make links footnotes instead of hotlinks:
\DeclareRobustCommand{\href}[2]{#2\footnote{\url{#1}}}
\definecolor{default-linkcolor}{HTML}{A50000}
\definecolor{default-filecolor}{HTML}{A50000}
\definecolor{default-citecolor}{HTML}{4077C0}
\definecolor{default-urlcolor}{HTML}{4077C0}

\hypersetup{
  pdftitle={lernOS Expert Debriefing Leitfaden},
  pdfauthor={Simon Dückert; Silvia Roderus},
  pdflang={de-de},
  hidelinks,
  breaklinks=true,
  pdfcreator={LaTeX via pandoc with the Eisvogel template}}

\title{lernOS Expert Debriefing Leitfaden}
\usepackage{etoolbox}
\makeatletter
\providecommand{\subtitle}[1]{% add subtitle to \maketitle
  \apptocmd{\@title}{\par {\large #1 \par}}{}{}
}
\makeatother
\subtitle{Wissen von Expert:innen nachhaltig bewahren}
\author{Simon Dückert \and Silvia Roderus}
\date{Version 3.1 (18.11.2025)}


%
% for the background color of the title page
%
\usepackage{pagecolor}
\usepackage{afterpage}
\usepackage[margin=2.5cm,includehead=true,includefoot=true,centering]{geometry}

%
% break urls
%
\PassOptionsToPackage{hyphens}{url}

%
% When using babel or polyglossia with biblatex, loading csquotes is recommended
% to ensure that quoted texts are typeset according to the rules of your main language.
%
\usepackage{csquotes}

%
% captions
%
\definecolor{caption-color}{HTML}{777777}
\usepackage[font={stretch=1.2}, textfont={color=caption-color}, position=top, skip=4mm, labelfont=bf, singlelinecheck=false, justification=raggedright]{caption}
\setcapindent{0em}

%
% blockquote
%
\definecolor{blockquote-border}{RGB}{221,221,221}
\definecolor{blockquote-text}{RGB}{119,119,119}
\usepackage{mdframed}
\newmdenv[rightline=false,bottomline=false,topline=false,linewidth=3pt,linecolor=blockquote-border,skipabove=\parskip]{customblockquote}
\renewenvironment{quote}{\begin{customblockquote}\list{}{\rightmargin=0em\leftmargin=0em}%
\item\relax\color{blockquote-text}\ignorespaces}{\unskip\unskip\endlist\end{customblockquote}}

%
% Source Sans Pro as the default font family
% Source Code Pro for monospace text
%
% 'default' option sets the default
% font family to Source Sans Pro, not \sfdefault.
%
% Note that the font has been officially renamed to `Source Sans 3`, and
% the version provided by the `sourcesanspro` package is slightly outdated.
% You can install the newer version locally and use it, for example, with
% `mainfont: "Source Sans 3"` in the YAML metadata (requires XeTeX or LuaTeX).
%
\ifnum 0\ifxetex 1\fi\ifluatex 1\fi=0 % if pdftex
    \usepackage[default]{sourcesanspro}
  \usepackage{sourcecodepro}
  \else % if not pdftex
    \usepackage[default]{sourcesanspro}
  \usepackage{sourcecodepro}

  % XeLaTeX specific adjustments for straight quotes: https://tex.stackexchange.com/a/354887
  % This issue is already fixed (see https://github.com/silkeh/latex-sourcecodepro/pull/5) but the
  % fix is still unreleased.
  % TODO: Remove this workaround when the new version of sourcecodepro is released on CTAN.
  \ifxetex
    \makeatletter
    \defaultfontfeatures[\ttfamily]
      { Numbers   = \sourcecodepro@figurestyle,
        Scale     = \SourceCodePro@scale,
        Extension = .otf }
    \setmonofont
      [ UprightFont    = *-\sourcecodepro@regstyle,
        ItalicFont     = *-\sourcecodepro@regstyle It,
        BoldFont       = *-\sourcecodepro@boldstyle,
        BoldItalicFont = *-\sourcecodepro@boldstyle It ]
      {SourceCodePro}
    \makeatother
  \fi
  \fi

%
% heading color
%
\definecolor{heading-color}{RGB}{40,40,40}
% By default, the KOMA-Script classes will typeset sectioning headings in
% sans-serif. Use the normal body font for headings.
\addtokomafont{disposition}{\normalfont\color{heading-color}\bfseries}

%
% variables for title, author and date
%
\usepackage{titling}
\title{lernOS Expert Debriefing Leitfaden}
\author{Simon Dückert, Silvia Roderus}
\date{Version 3.1 (18.11.2025)}

%
% tables
%

\definecolor{table-row-color}{HTML}{F5F5F5}
\definecolor{table-rule-color}{HTML}{999999}

%\arrayrulecolor{black!40}
\arrayrulecolor{table-rule-color}     % color of \toprule, \midrule, \bottomrule
\setlength\heavyrulewidth{0.3ex}      % thickness of \toprule, \bottomrule
\renewcommand{\arraystretch}{1.3}     % spacing (padding)


%
% remove paragraph indentation
%
\setlength{\parindent}{0pt}
\setlength{\parskip}{6pt plus 2pt minus 1pt}
\setlength{\emergencystretch}{3em}  % prevent overfull lines

%
%
% Listings
%
%


%
% header and footer
%
\usepackage[headsepline,footsepline]{scrlayer-scrpage}

\newpairofpagestyles{eisvogel-header-footer}{
  \clearpairofpagestyles
  \ihead*{lernOS Expert Debriefing Leitfaden}
  \chead*{}
  \ohead*{Version 3.1 (18.11.2025)}
  \ifoot*{Simon Dückert, Silvia Roderus}
  \cfoot*{}
  \ofoot*{\thepage}
  \addtokomafont{pageheadfoot}{\upshape}
}
\pagestyle{eisvogel-header-footer}



%
% Define watermark
%

\begin{document}

\begin{titlepage}
\newgeometry{left=6cm}
\definecolor{titlepage-color}{HTML}{ff6600}
\newpagecolor{titlepage-color}\afterpage{\restorepagecolor}
\newcommand{\colorRule}[3][black]{\textcolor[HTML]{#1}{\rule{#2}{#3}}}
\begin{flushleft}
\noindent
\\[-1em]
\color[HTML]{ffffff}
\makebox[0pt][l]{\colorRule[ffffff]{1.3\textwidth}{4pt}}
\par
\noindent

{
  \setstretch{1.4}
  \vfill
  \noindent {\huge \textbf{\textsf{lernOS Expert Debriefing Leitfaden}}}
    \vskip 1em
  {\Large \textsf{Wissen von Expert:innen nachhaltig bewahren}}
    \vskip 2em
  \noindent {\Large \textsf{Simon Dückert, Silvia Roderus}}
  \vfill
}


\textsf{Version 3.1 (18.11.2025)}
\end{flushleft}
\end{titlepage}
\restoregeometry
\pagenumbering{arabic}

% don't generate the default title
% \maketitle


{
\setcounter{tocdepth}{3}
\tableofcontents
\newpage
}
\section{Vorwort}\label{vorwort}

Viele Organisationen sehen sich in der Wissensgesellschaft des 21.
Jahrhundert mit einer ganz besonderen Situation konfrontiert. Durch
Globalisierung, Digitalisierung sowie schnelle technologische und
wissenschaftliche Entwicklung sehen sie sich einem bisher nicht da
gewesenen Wettbewerbsdruck gegenübergestellt. Auch der Charakter der
Arbeit hat sich geändert. Bestand die Arbeit 1930 noch zu ca. 80\% aus
manueller Routinearbeit, so dominiert heute die Wissensarbeit.
Gleichzeitig führt die Alterung der Gesellschaft durch den
\href{https://de.wikipedia.org/wiki/Demografischer_Wandel}{demografischen
Wandel},
\href{https://de.wikipedia.org/wiki/Fachkr\%C3\%A4ftemangel}{Fachkräftemangel}
und \href{https://de.wikipedia.org/wiki/Jobrotation}{Jobrotationen} zu
Verlust von wertvollem
\href{https://de.wikipedia.org/wiki/Erfahrungswissen}{Erfahrungswissen}
und Schwächung der Innovationskraft. Deswegen sollten Unternehmen
systematische Prozesse der Wissenssicherung etablieren. Mit dem Expert
Debriefing erlernst Du eine Methode zur systematischen Bewahrung des
\href{https://de.wikipedia.org/wiki/Wissen}{Wissens} von
\href{https://de.wikipedia.org/wiki/Experte}{Experten}.

Die Methode Expert Debriefing dient dazu, das Wissen eines
ausscheidenden oder wechselnden Experten der Organisation zu bewahren
sowie dem Experten Wertschätzung für seine Leistungen zu zeigen. Um
diesen Prozess systematisch zu gestalten, wird ein Expert Debriefing von
einem ausgebildeten Moderator begleitet. Expert Debriefings können bei
Fach- und Führungskräften gleichermaßen sowie in allen funktionalen
Bereichen (Marketing, Vertrieb, Entwicklung, Produktion, Service)
durchgeführt werden. Der Begriff des Experten ist hierbei immer relativ
in Beziehung zu einer Bezugsgruppe, den Laien, zu verstehen.

\pandocbounded{\includegraphics[keepaspectratio]{./images/expert-debriefing-rollen.png}}

Da jeder Experte mit seinem Wissen und seiner Neigung zu Werkzeugen der
Wissensbewahrung unterschiedlich sind, gibt es keinen
One-Size-Fits-All-Ansatz geben. Der Expert Debriefing Referenzprozess
ist deshalb so aufgebaut, dass zunächst in einer persönlichen
Wissenslandkarte des Experten Überblick über alle möglichen
Wissensgebiete aufgebaut wird, um dann auf Basis dieser Übersicht die
wichtigsten Maßnahmen auszuwählen. Zur Auswahl der Maßnahmen steht die
Expert Debriefing Toolbox mit praxiserprobten Tools und Methoden zur
Verfügung.

Mit diesem Leitfaden lernst du alle Grundlagen des Expert Debriefings
kennen. Mit den Übungen (Katas) aus dem Lernpfad lernst du den ganzen
Prozess kennen.

\section{Über lernOS}\label{uxfcber-lernos}

\textbf{lernOS} ist ein
\href{https://de.wikipedia.org/wiki/Offenes_System}{offenes System} für
\href{https://de.wikipedia.org/wiki/Lebenslanges_Lernen}{Lebenslanges
Lernen} und
\href{https://de.wikipedia.org/wiki/Lernende_Organisation}{Lernende
Organisationen}. Die Funktionsweise von lernOS wird in
\href{https://opendefinition.org/od/2.1/de/}{offen} verfügbaren
\textbf{Leitfäden} beschrieben. lernOS kann ganz einfach als
\textbf{Einzelperson}, im \textbf{Team} oder in der gesamten
\textbf{Organisation} praktiziert werden.

Startest du als Einzelperson, empfehlen wir dir, gemeinsam in einem
\textbf{Circle} (4-5 Personen) oder zumindest in einem
\textbf{Lerntandem} (2 Personen) zu starten. Weitere Informationen
findest du auf der Seite \href{http://lernos.org}{lernos.org}.
Mitstreiter findest du ganz einfach in der
\href{https://community.cogneon.de}{lernOS Community CONNECT}. Dort gibt
es auch den
\href{https://community.cogneon.de/c/lernos/lernos-circlefinder/}{lernOS
Circlefinder} als Marktplatz für Circle-Angebote und -Gesuche.

\textbf{KEEP CALM \& LEARN ON!}

\section{Grundlagen}\label{grundlagen}

\subsection{Geschichte des Expert
Debriefings}\label{geschichte-des-expert-debriefings}

Die Methode Expert Debriefing ist ursprünglich unter der Bezeichnung
``Knowledge Engineering'' von der Forschungsgruppe Wissenserwerb
(FORWISS) vom Lehrstuhl für Künstliche Intelligenz der Universität
Erlangen-Nürnberg bei Prof.~Stoyan entwickelt worde.

Folgende Liste zeigt wichtige Meilensteine in der Entwicklung der
Methode Expert Debriefing (ausführlich unter
\href{https://wiki.cogneon.de/Geschichte_der_Cogneon_Methode_Expert_Debriefing}{Geschichte
der Cogneon Methode Expert Debriefing}):

\begin{itemize}
\tightlist
\item
  \textbf{1990er:} EWITA-Projekte bei Audi zum Wissenstransfer von
  erfahrenen auf neue Mitarbeiter.
\item
  \textbf{Ab 1999:} Entwicklung eines expliziten „Expert Debriefing
  Prozesses`` für Einzelexperten.
\item
  \textbf{2002:} Erste diehMultiplikatoren-Projekte und -Schulungen bei
  Volkswagen, dort Wissensstafette genannt; Gewinn des Preises
  \href{https://www.managerseminare.de/ms_News/VW-Coaching-Ausgezeichnetes-Wissensmanagement,155460}{Wissensmanager
  des Jahres 2006}.
\item
  \textbf{2005:} Expert Debriefing Einführung bei Schaeffler, dort
  Kopplung mit Wiki und Wissensgemeinschaften (Communities of Practice).
\item
  \textbf{2007:} Flächendeckende Verbreitung von Expert Debriefing durch
  offenes Schulungsangebot.
\item
  \textbf{2015:} Diehl erreicht mit Expert Debriefing
  \href{https://www.humanresourcesmanager.de/news/wissen-der-generationen-verbinden.html}{Platz
  drei HR Excellence Award}.
\item
  \textbf{2018:} Die Schulungsunterlage der Expert Debriefing Ausbildung
  wird als Teil der \href{https://lernos.org}{lernOS Leitfäden} offen
  verfügbar.
\end{itemize}

Das Expert Debriefing wird auch unter anderen Namen wie Wissensstafette,
Transferwerk, Wissensstaffel, Keep Experience und strukturierte
Wissensweitergabe in verschiedenen Unternehmen angewendet. Seit Juni
2011 gibt es eine jährliches
\href{https://community.cogneon.de/c/events/edmod-meetup}{Expert
Debriefing Moderatoren Meetup}, in der sich Moderator*innen mit
langjähriger Anwendungserfahrung regelmäßig austauschen.

\subsection{Warum ist Wissensbewahrung heute
wichtig?}\label{warum-ist-wissensbewahrung-heute-wichtig}

Wir befinden uns in der Transformation von Industrie- in
Wissensgesellschaft. Dieser Übergang ist nicht so deutlich zu spüren,
wie der Übergang von Agrar- in Industriegesellschaft, macht sich aber
gerade bei den wichtigsten Produktionsfaktoren deutlich bemerkbar:

\begin{quote}
Nicht Arbeit, nicht Kapital, nicht Land und Rohstoffe sind die
Produktionsfaktoren, die heute in unserer Gesellschaft zählen, sondern
das Wissen der Mitarbeiter in den Unternehmen - Peter F. Drucker (1909 -
2005)
\end{quote}

An vier Trends kann man die steigende Bedeutung der Wissensbewahrung
beispielhaft zeigen:

\begin{enumerate}
\def\labelenumi{\arabic{enumi}.}
\tightlist
\item
  \textbf{Wissensgesellschaft} - Wissen macht den Unterschied
\item
  \textbf{Demografischer Wandel} -- wenn Erfahrung in Rente geht
\item
  \textbf{Fachkräftemangel} -- wenn immer weniger immer mehr machen
  müssen
\item
  \textbf{Fluktuation} -- neue Jobrealitäten
\end{enumerate}

\subsubsection{Wissensgesellschaft}\label{wissensgesellschaft}

In der Drei-Sektoren-Hypothese von Jean Fourastié beschreibt er, dass
sich der Schwerpunkt der wirtschaftlichen Tätigkeit zunächst vom
primären Wirtschaftssektor (Rohstoffgewinnung), auf den sekundären
Sektor (Rohstoffverarbeitung) und anschließend auf den tertiären Sektor
(Dienstleistung) verlagert.

Der erste Sektor wird auch als Agrargesellschaft bezeichnet. Der Fokus
lag hier in der forst- und landwirtschaftlichen Erzeugung. Mit dem
technischen Fortschritt und der industriellen Revolution erfolgt auch
der Umschwung in den sekundären Sektor, der Industriegesellschaft. Hier
stand die industrielle Produktion im Vordergrund. Im Laufe der Zeit
wandelte sich der Fokus zur Erbringung von Dienstleistungen und damit
gesellschaftlich zum tertiären Sektor -- der
Dienstleistungsgesellschaft.

In der Dienstleistungsgesellschaft erfolgte immer stärker die
Verlagerung auf immaterielle Güter. „Wissen`` wurde vierter
Produktionsfaktor und gewinnt immer mehr an Bedeutung. Mit mindestens 60
\% ist es heute in vielen Unternehmen für die Gesamtwertschöpfung des
Unternehmens verantwortlich. Damit einhergehend wuchs ständig das Feld
der wissensintensiven Tätigkeiten und Berufe. Gehörten 1930 noch 83\%
dem Arbeitstyp „Arbeiter`` (Produktion, Rohstoffgewinnung,
Landwirtschaft) an, waren es 2000 nur noch 15\%. Demgegenüber stiegen im
gleichen Zeitraum der Arbeitstyp „Service-Arbeiter`` (Büro Jobs, Handel,
Basisdienstleistungen) von 6\% auf 15 \% an und der Arbeitstyp
„Wissensarbeiter`` (Beratung, Coaching, F\&E, Management) von 8\% auf
25\%.

Wissen wird damit zur strategischen Ressource in Produkten und
Dienstleistungen. Das Wissen ist vernetzt, dezentral und
interdisziplinär. Die effektive Nutzung des Wissens wird zum
entscheidenden Wettbewerbsfaktor, die Gesellschaft ist damit in der
Wissensgesellschaft angekommen.

\subsubsection{Demografischer Wandel}\label{demografischer-wandel}

Neben diesem ökonomischen Trend gibt es noch einen weiteren
gesellschaftlichen Trend, den des demografischen Wandels. Der
demografische Wandel beschreibt eine älter werdenden oder alternde
Gesellschaft. Dieser Wandel hat große Auswirkungen auf alle
gesellschaftlichen Bereiche. Bis 2030 sinkt die Bevölkerung um fünf
Millionen, es gibt 17\% weniger Kinder und Jugendliche und 33\% mehr
Bürger über 65 Jahren, die Gruppe der Personen im erwerbsfähigen Alter
schrumpft um 15\%.

Viele der Mitarbeiter, die heute oder in den nächsten Jahren in Rente
gehen, haben die Unternehmen, in denen sie beschäftig sind, mit
aufgebaut. Sie haben große Errungenschaften erlebt, aber auch die ein
oder andere Katastrophe. Damit haben sie einen großen Erfahrungsschatz
angesammelt und dieser steht nun kurz davor, die Unternehmen, gemeinsam
mit ihren Wissensträgern, zu verlassen. Natürlich wurden im Laufe der
Jahre auch Dokumentationen erstellt.

Aber häufig stellen sich die Fragen: Wo ist was dokumentiert? Was ist
noch für die Zukunft wichtig und wurde bislang noch nicht festgehalten?
Wie können Erfahrungen bewahrt und transferiert werden? Was bedeutet es
für Unternehmen, wenn einerseits immer höhere Anforderungen durch die
Wissensgesellschaft an sie herangetragen werden und andererseits immer
weniger Mitarbeiter zur Verfügung stehen?

\subsubsection{Fachkräftemangel}\label{fachkruxe4ftemangel}

Auch wenn die Hörsäle aktuell eher überfüllt sind, werden sie
tendenziell leerer werden. Schon heute mangelt es an Fachkräften in der
Wirtschaft. Rund 11 Mrd. Euro sind es allein in der IT-Branche, die
durch Wissens- und Kompetenzverlust entstehen. Diese Lücke entsteht u.a.
auch dadurch, dass es nicht genügend Fachkräfte für die Nachbesetzung
gibt bzw. die Nachbesetzung sich sehr lange hinzieht (durchschnittliche
Vakanzzeiten von 55 Tagen über alle Positionen hinweg und bis zu 90
Tagen bei technischen Berufen). Was kann getan werden, wenn immer
weniger immer mehr machen müssen? Wie kann sich ein Unternehmen vor dem
Wissensverlustrisiko besser schützen?

\subsubsection{Fluktuation}\label{fluktuation}

Das Mitarbeiterengagement ist seit 2004 gesunken. 33 \% der Arbeitnehmer
in Deutschland denken ernsthaft darüber nach, das Unternehmen zu
wechseln. Dabei liegt die Wechselbereitschaft der bis 24-Jährigen liegt
bei 50 \% und die der 25- bis 34-Jährigen um 40 \%. Das ist die Gruppe
der jungen Talente, um die im „War for talents`` im Zuge des
Fachkräftemangels so begehrlich gekämpft wird.

Das Arbeitsmodell „Arbeit in einem Unternehmen bis zur goldenen Uhr``
hat ausgedient. Gelebte Work-Life-Balance, Individualisierung und
Selbstverwirklichung sind aktuelle Trends. Damit steht auch fest, dass
es einen anderen Umgang mit dem Thema Wissen bzw. Wissensbewahrung in
den Unternehmen geben muss. Wie stellt sich ein Unternehmen auf diese
neuen Jobrealitäten ein? Welche Möglichkeiten der Wissensbewahrung gibt
es?

\subsection{Implizites und explizites
Wissen}\label{implizites-und-explizites-wissen}

Das Wort \href{https://de.wikipedia.org/wiki/Wissen}{Wissen} stammt von
althochdeutsch „wizzan`` bzw. der indogermanischen Perfektform „woida``
und bedeutet „ich habe gesehen``, somit auch „ich weiß``. Im
Wissenstransfer kann man das sehr leicht beobachten. Kleine Kinder
lernen am Anfang durch beobachten und nachmachen. Wissen wird also
transferiert, indem eine Beobachtung erfolgt.

Dieser Mechanismus kann auf die Welt der Erwachsenen übertragen werden.
Person A macht etwas und Person B beobachtet Person A dabei. In der
nächsten Stufe spricht Person A mit Person B über etwas. Hier wird der
Transfer um die Ebene der Sprache erweitert. Es können auch beide Stufen
durchlaufen werden, d.h. es wird erst etwas beobachtet und anschließen
oder dabei mit einander gesprochen. Da hier die direkte Interaktion
zwischen den Wissensträger im Vordergrund steht, spricht man auch von
der \textbf{Wissenskommunikation}.

Die Person A könnte aber auch etwas dokumentierten und dieses Dokument
ablegen. Wenn Person B dann einen bestimmten Inhalt sucht, findet sie
das Dokument. Person B kann es dann lesen und verstehen. Da in dieser
Form des Wissenstransfers das Erstellen (Dokumentieren) von
Wissensobjekten im Vordergrund steht, spricht man auch von der
\textbf{Wissensdokumentation}. Ein Wissensobjekt kann dabei ein Text in
einem Wiki oder Dokumentenmanagement-System sein oder auch eine Audio-
bzw. Video-Datei.

\pandocbounded{\includegraphics[keepaspectratio]{./images/Wissenstransfer.png}}

Wissenskommunikation und Wissensdokumentation kommen in einem
Unternehmen nie in Reinform vor. Es handelt sich immer um Mischformen.
Dennoch sollte in der Wissensmanagementstrategie entschieden werden, ob
der Fokus eher auf Kodifizierung ausgerichtet ist, also die
Dokumentation in Wiki's, Dokumentenmanagement Systeme, etc. im
Vordergrund steht oder ob sie eher auf eine Personalisierung
ausgerichtet ist, d.h. die direkte Kommunikation über Lessons Learned
Workshops, Wissensgemeinschaften, Lerntandems, etc.

Wichtig ist beim Wissenstransfer die Unterscheidung zwischen implizitem
und explizitem Wissen. \textbf{Explizites Wissen} ist dem Experten
bewusst und kann durch diesen in Wort gefasst werden. Explizites Wissen
ist damit der Dokumentation gut zugänglich (dokumentiertes Wissen).
\textbf{Implizites Wissen} ist dem Experten dagegen nicht bewusst. Er
handelt zwar unbewusst auf Basis dieses Wissen, kann es für den
Nachfolger aber nicht beschreiben oder gar dokumentieren. Das implizite
Wissen eines Experten bewusst und sichtbar zu machen, es zu
strukturieren und zu dokumentieren ist Kernaufgabe des Expert
Debriefings.

\textbf{Beispiel:} Ein Experte bekommt täglich viele Anfragen von
verschiedenen Personen. Viele der Anfragen wiederholen sich.

\begin{itemize}
\tightlist
\item
  \textbf{Lösung mit Fokus auf Wissenskommunikation} - Er bekommt die
  Fragen per Telefon oder die Leute kommen direkt zu ihm. Er bespricht
  mit jedem einzelnem die Fragen. Jeder Einzelne ist sehr zufrieden, es
  herrscht eine hohe Bindung an den Experten. Der Experte benötigt dafür
  viel Zeit.
\item
  \textbf{Lösung mit Fokus auf Wissensdokumentation} - Der Experte
  erstellt eine Liste der Häufig gestellten Fragen (FAQ-Liste) mit
  entsprechenden Antworten. Alle eingehenden Fragen verweist er auf die
  FAQ Liste. Er beschäftigt sich nur mit Anfragen, die nicht in der
  FAQ-Liste stehen. Die FAQ-Liste wird ständig durch ihn ergänzt. Das
  Abweisen auf eine FAQ-Liste kann für viele erst einmal unpersönlich
  vorkommen. Dadurch entsteht nicht so eine hohe Bindung an den
  Experten. Der Experte hat nach einem anfänglichen Mehraufwand später
  weniger Aufwand in der Beantwortung von Fragen. Das Unternehmen hat
  zudem eine Wissensbasis, sollte der Experte einmal ausfallen.
\end{itemize}

Beide Lösungen sind möglich -- jedes Unternehmen muss für sich
entscheiden, welches der richtigere Weg ist. Genauso verhält es sich mit
den Wissensbewahrungsmethoden. Häufig könnten verschiedene Methoden zum
Einsatz kommen. Entscheidend für die Auswahl ist Reifegrad der
Organisation und die strategische Wissensmanagementausrichtung.

\subsection{Erfolgsfaktoren für Expert
Debriefings}\label{erfolgsfaktoren-fuxfcr-expert-debriefings}

Da es die Methode Expert Debriefing schon seit den 1990er Jahren gibt,
wurden bereits viele Erfahrungen in unterschiedlichen Branchen und
Unternehmensgrößen gesammelt. Aus diesen Erfahrungen können folgende
Erfolgsfaktoren für das Gelingen eines Expert Debriefings benannt
werden:

\begin{itemize}
\tightlist
\item
  \textbf{Die Teilnahme am Expert Debriefing ist freiwillig:} Die
  ausdrückliche Bereitschaft des Experten sein Wissen zu teilen ist
  existenziell. Die Motivation des Experten, dies zu tun, ist von den
  Rahmenbedingungen für die Durchführung des Expert Debriefings
  abhängig. Gehen Experten in den Ruhestand oder liegt eine Kündigung
  des Experten vor und er ist dem Unternehmen wohlgesonnen (z.B. wenn
  der Experte aus privaten Gründen umzieht), ist eine hohe Motivation
  vorhanden. Wird unternehmensseitig z.B. im Rahmen von
  Umstrukturierungsmaßnahmen eine Kündigung ausgesprochen, kann das
  Expert Debriefing in die Aufhebungs- oder Sozialverhandlungen mit
  eingebunden werden. Bei zerrütteten Arbeitsverhältnissen und deren
  Beendigung ist hingegen kaum noch eine Motivation vorhanden.
\item
  \textbf{Vorgesetzte stehen hinter dem Prozess, lassen aber Freiräume
  zu:} Der Vorgesetzte verantworte die Ressourcen (Zeit des Experten/
  ggf. der Nachfolger, Budget, Ort) für die Prozessdurchführung und ist
  maßgeblich an der Festlegung der Zielinhalte beteiligt. Dass sich der
  Vorgesetzte dem Gelingen des Expert Debriefings verpflichtet fühlt
  (Commitment) ist notwendig, damit die Transfermaßnahmen im
  Prozessverlauf reibungslos umgesetzt werden können. Sind Auftraggeber
  und Vorgesetzte nicht identisch, so sind beide in den Prozess zu
  involvieren. Unstimmigkeiten sind von den beiden Parteien direkt zu
  klären. Die Erhebung der persönlichen Wissenslandkarte sollte ohne den
  Vorgesetzten erfolgen, da der Experte dann i.d.R. offener agiert
  (gerade im Bereich der Lessons Learned).
\item
  \textbf{Transparenz schaffen und Erwartungshaltungen der Beteiligten
  abholen:} Transparenz (Warum das Expert Debriefing erfolgt, Was sind
  die Inhalte und was passiert mit den Ergebnissen) im Vorfeld erzeugen.
  Der abstrakte Prozess kann durch Fallbeispiele konkretisiert werden.
  Ein Vorgespräch mit Auftraggeber und/oder Vorgesetzen, sowie Experten
  und falls vorhanden dem Nachfolger nimmt Erwartungen auf, klärt auf
  und nimmt dadurch Ängste.
\item
  \textbf{Den Prozess professionell und auf Augenhöhe moderieren:}
  Respektvoller und vertrauensvoller Umgang zwischen Moderator, Experten
  und Nachfolger ermöglichen das Erheben eines breiten Wissensspektrums
  und das erschließen „impliziten`` Wissens. Explizites Wissen sollte
  explizit übergeben werden (vorhandene Dokumentationen). Implizites
  Wissen sollte externalisiert werden (z.B. durch Erfahrungsgeschichten
  und Podcasts). Der Moderator sollte dabei eine fachliche Augenhöhe zum
  Experten haben, d.h. dass Begrifflichkeiten bekannt sind. Dazu muss
  sich der Moderator ggf. im Vorfeld in die Thematik einlesen.
\item
  \textbf{Den Experten durch Wertschätzung motivieren:} Eine
  Wertschätzung des Experten ist ein hoher Motivationsfaktor für die
  Bereitschaft, Wissen zu teilen.
\item
  \textbf{Den Prozess durch den Moderator strukturieren und
  kontrollieren:} Das Aufsetzen des Prozesses mit Vorgespräch,
  persönlicher Wissenslandkarte und Ableitung eines Maßnahmenplans durch
  den Moderator gibt die notwendige Struktur. Die Umsetzung des
  Maßnahmenplans durch das Transfertandem sollte über regelmäßigen
  Reviews „kontrolliert`` werden. Der Moderator hat hierbei keine
  inhaltliche Verantwortung, ist aber für die Prozessabwicklung
  verantwortlich. Werden Maßnahmen nicht termingerecht umgesetzt, muss
  er eingreifen und ggf. an den Vorgesetzten eskalieren.
\end{itemize}

\subsection{Expert Debriefing
Referenzprozess}\label{expert-debriefing-referenzprozess}

Der Expert Debriefing Referenzprozess besteht aus sechs Schritten, die
i.d.R. über einen Zeitraum von 3-6 Monaten ablaufen. In diesem Zeitraum
steht oft ein Zeitkontingent von 5-10\% der Arbeitszeit zur Verfügung.
Die volle Arbeitszeit von Experte und Nachfolger zu nutzen ist nicht
möglich, wenn der Experte noch in Tätigkeiten eingebunden ist.

\begin{figure}
\centering
\pandocbounded{\includegraphics[keepaspectratio,alt={Expert Debriefing Referenzprozess}]{images/Prozess-Expert-Debriefing.png}}
\caption{Expert Debriefing Referenzprozess}
\end{figure}

Der Prozess beinhalte folgende Aufgaben für den Moderator:

\begin{enumerate}
\def\labelenumi{\arabic{enumi}.}
\tightlist
\item
  \textbf{Vorgespräch führen:} Das Vorgespräch dient dazu, dem
  Auftraggeber und dem Experten den Zweck, die Vorgehensweise und die
  Ergebnisse eines Expert Debriefings aufzuzeigen, von den Beteiligten
  einen Überblick über die Situation und die Rahmenbedingungen zu
  erhalten, durch den Auftraggeber den Fokus für das Expert Debriefing
  festlegen zu lassen und die weiteren konkreten Schritte zu planen. Ein
  Expert Debriefing sollte immer auf freiwilliger Teilnahme des Experten
  basieren, da man Wissensteilung nicht erzwingen kann.
\item
  \textbf{Persönliche Wissenslandkarte aufbauen:} Die persönliche
  Wissenslandkarte dient dazu, einen systematischen und vollständigen
  Überblick über das gesamte in Bezug auf eine Stelle relevante Wissen
  herzustellen. Die Wissenslandkarte enthält i.d.R. aber keine
  ausführliche Wissensdokumentation (Struktur, kein Inhalt; ``The Map is
  not the Territory''). Die persönliche Wissenslandkarte kann somit als
  Wissenslandkarte einer Stelle betrachtet werden. Die persönliche
  Wissenslandkarte beinhaltet im Gegensatz zu einer einfachen Mind Map
  eine vorstrukturierte erste Ebene (Arbeitshistorie, Aufgaben und
  Wissensgebiete), um systematisch das Gedächtnis und damit das
  implizite Wissen des Experten zu aktivieren (Episoden-, prozedurales
  und deklaratives Gedächtnis).
\item
  \textbf{Maßnahmen ableiten:} Die Ableitung eines Maßnahmen-Plans dient
  dazu, geeignete Maßnahmen zur Wissensbewahrung zu identifizieren, sie
  sowohl durch Experten als auch durch den Nachfolger priorisieren zu
  lassen und anschließend alle Maßnahmen zu terminieren.
\item
  \textbf{Feedback einholen:} Das Einholen des Feedbacks dient dazu,
  einen möglichst objektiven Überblick über die notwendigen Maßnahmen
  zur Wissensbewahrung zu erhalten und dem Auftraggeber (ggf. weiteren
  Kollegen) die Möglichkeit zu geben, in den Maßnahmen-Plan korrigierend
  einzugreifen.
\item
  \textbf{Maßnahmen begleiten:} Die Durchführung der im Maßnahmen-Plan
  festgelegten Maßnahmen ist der Kern des Expert Debriefings und dient
  der Wissensbewahrung durch Wissensidentifikation, Wissensdokumentation
  oder Wissenskooperation. Ziel ist, dass Experte und Nachfolger
  möglichst viele Maßnahmen in Eigenregie und in ihren Arbeitsalltag
  integriert durchführen. Der Moderator hat hier zwei Rollen: 1.
  Projektleiter: er wacht darüber, dass die im Maßnahmen-Plan
  festgelegten Maßnahmen durchgeführt werden. 2. Unterstützer: in
  1:NN-Szenarien (Nachfolger noch nicht bekannt) oder bei komplexeren
  Maßnahmen unterstützt der Moderator bei konkreten Maßnahmen.
\item
  \textbf{Reflexion moderieren:} Die Reflexion dient der
  Umsetzungskontrolle sowie der kontinuierlichen Verbesserung der
  Methode Expert Debriefing. Darüber hinaus sollen
  Verbesserungspotentiale in der Organisation identifiziert werden, die
  den Einsatz der Methode Expert Debriefing langfristig überflüssig
  machen können. Optional können Vorgesetzter und Nachfolger ca. drei
  Monate nach Abschluss des Expert Debriefings befragt werden, ob die
  Übergabe durch die Methode besser war, als ohne und ob es zu größeren
  Überraschungen gekommen ist.
\end{enumerate}

Die einzelnen Aufgaben werden in den folgenden Kapiteln im Detail
erläutert.

\subsubsection{Vorgespräch führen}\label{vorgespruxe4ch-fuxfchren}

Das \textbf{Vorgespräch} dient dazu, dem Auftraggeber, dem Experten (E)
und dem Nachfolger (N), sofern schon vorhanden, den Zweck, die
Vorgehensweise und die Ergebnisse eines Expert Debriefings aufzuzeigen.
Der Moderator (M) erhält von den Beteiligten einen Überblick über die
Situation und die Rahmenbedingungen, lässt durch den Auftraggeber den
Fokus für das Expert Debriefing festlegen und plant die weiteren
konkreten Schritte.

Damit dient das Vorgespräch auch als Auftragsklärung, d.h. der gesamte
Rahmen des Expert Debriefings für Inhalt, Zeitraum (Dauer) und Aufwand
(Verfügbarkeit des Experten in dem Zeitraum) wird vereinbart. Besonders
wichtig ist hier das Commitment zwischen Auftraggeber (A)/Vorgesetzter
(V) und Experten (E). Sollten im Prozessverlauf Schwierigkeiten
auftreten (z.B. keine Zeit für die Durchführung von
Wissensbewahrungsmethoden), kann auf diese ``Vereinbarung''
zurückgegriffen werden. In einigen Unternehmen unterschreiben die
Beteiligten das Protokoll, womit eine stärkere Bindung erzielt wird.

\textbf{Hinweis:} es wird davon ausgegangen, dass Auftraggeber und
Vorgesetzter identisch sind. Ist das nicht der Fall, muss der
Personenkreis erweitert werden.

{\def\LTcaptype{none} % do not increment counter
\begin{longtable}[]{@{}lcccc@{}}
\toprule\noalign{}
Prozessschritt & E & N & V & M \\
\midrule\noalign{}
\endhead
\bottomrule\noalign{}
\endlastfoot
Vorgespräch führen & X & & X & X \\
Persönliche Wissenslandkarte aufbauen & X & X & & X \\
Maßnahmen ableiten & X & X & & X \\
Feedback einholen & & & X & \\
Maßnahmen begleiten & X & X & & X \\
Reflexion moderieren & X & X & X & X \\
\end{longtable}
}

\textbf{Vorgehensweise:}

\begin{enumerate}
\def\labelenumi{\arabic{enumi}.}
\tightlist
\item
  Vorstellungsrunde mit Klärung der Rollen.
\item
  Die Methode Expert Debriefing anhand des Referenzprozesses erklären
  und offene Fragen beantworten.
\item
  Situation und Rahmenbedingungen erfragen, Entscheidungen treffen und
  dokumentieren.
\item
  Termine für Aufbau persönliche Wissenslandkarte, Ableitung
  Maßnahmen-Plan, Feedback, evtl. Begleitung der Maßnahmen sowie die
  Reflexion vereinbaren.
\item
  Protokoll an Teilnehmer des Vorgesprächs verschicken.
\end{enumerate}

\textbf{Ressourcen und Hilfsmittel:}

\begin{itemize}
\tightlist
\item
  Vorlage \textbf{Protokoll Vorgespräch} in der Expert Debriefing
  Toolbox.
\end{itemize}

\textbf{Tipps und Tricks:}

\begin{itemize}
\tightlist
\item
  Beamer für Präsentation und Protokoll nutzen, damit alle Beteiligten
  die Dokumentation sehen können.
\item
  Ggf. erst mit dem Vorgesetzten alleine sprechen (Fokus des
  Debriefings) und dann in großer Runde.
\item
  Der Auftraggeber muss nicht der Vorgesetzte sein. Wichtig ist hier die
  Involvierung und das Commitment des Vorgesetzten, da dieser später die
  Umsetzung ermöglichen muss.
\end{itemize}

\subsubsection{Persönliche Wissenslandkarte
aufbauen}\label{persuxf6nliche-wissenslandkarte-aufbauen}

Die \textbf{Persönliche Wissenslandkarte} dient dazu, einen
systematischen und vollständigen Überblick über das gesamte in Bezug auf
eine Stelle relevante Wissen herzustellen. Die persönliche
Wissenslandkarte kann somit als Wissenslandkarte einer Person und ihrer
Stelle betrachtet werden. Die persönliche Wissenslandkarte beinhaltet im
Gegensatz zu einer einfachen Mind Map eine vorstrukturierte erste Ebene,
um das Gedächtnis und damit das implizite Wissen des Experten
systematisch zu aktivieren:

\begin{enumerate}
\def\labelenumi{\arabic{enumi}.}
\tightlist
\item
  \textbf{Arbeitshistorie:} Episodengedächtnis, Aufhänger für
  Geschichten zu große Erfolgen und großen Katastrophen in der
  Vergangenheit (s.a.
  \href{https://de.wikipedia.org/wiki/Methode_der_kritischen_Ereignisse}{Methode
  der kritischen Ereignisse}).
\item
  \textbf{Aufgaben (und Rollen):} prozedurales Gedächtnis, mit dem
  Wissen über organisationale Prozesse und die eigenen Aufgaben darin.
\item
  \textbf{Wissensgebiete:} deklaratives Gedächtnis, Wissen zu welchen
  Fachgebieten benötigen die Aufgaben des Experten (oft auch Trends und
  Themen für die Zukunft).
\end{enumerate}

\textbf{ProTip:} Alle drei Bereiche können bereits Ideen für die
Ableitung des Maßnahmen-Plans beinhalten.

\begin{figure}
\centering
\pandocbounded{\includegraphics[keepaspectratio,alt={Beispiel Persönliche Wissenslandkarte}]{images/Persoenliche-Wissenslandkarte.png}}
\caption{Beispiel Persönliche Wissenslandkarte}
\end{figure}

\textbf{Vorgehensweise:}

\textbf{Abschnitt 1: Arbeitshistorie} in der Persönlichen
Wissenslandkarte im Bereich „Arbeitshistorie`` erheben. In einem
normalen Expert Debriefing nimmt das ca. 20\% der Zeit ein.

\begin{enumerate}
\def\labelenumi{\arabic{enumi}.}
\tightlist
\item
  In welchem Zeitraum könnte sich für den Nachfolger relevantes Wissen
  verbergen (z.B. Zeit beim aktuellen Arbeitgeber, x Jahre)?
\item
  Welche groben Phasen gab es in diesem Zeitraum? Den Zeitraum jeweils
  in Klammern angeben (z.B. Firma xy (1987-1993)).
\item
  Welche Aufgabenschwerpunkte und Ereignisse (positive und negative) gab
  es in den einzelnen Phasen?
\item
  Welche wichtigen Geschichten ranken sich rund um die Arbeitshistorie
  (Erfolge, Katastrophen, Lustiges, Peinliches etc.)?
\end{enumerate}

\textbf{ProTip:} für die Erarbeitung der Arbeitshistorie kann zur
Auflockerung auch die Methode
\href{https://www.christianhmeyer.de/die-lebenslinie-aus-der-vergangenheit-fuer-die-zukunft-lernen/}{Lebenslinie}
verwendet werden.

\textbf{Abschnitt 2: Aufgaben} in der persönlichen Wissenslandkarte im
Bereich „Aufgaben`` erheben. In einem normalen Expert Debriefing nimmt
das ca. 60\% der Zeit ein.

\begin{enumerate}
\def\labelenumi{\arabic{enumi}.}
\tightlist
\item
  Welche Aufgaben haben Sie? \textbf{Hinweis:} Format „Objekt + Verb``
  (z.B. „Schulung durchführen``). \textbf{Hinweis:} wichtige
  Ansprechpartner, Dokumente, Tipps \& Tricks können gleich mit erhoben
  werden, aber auf die Zeit achten.
\item
  Lassen sich die Aufgaben Ihrer Stelle sinnvoll in einzelne Rollen
  gruppieren(z.B. Führungskraft, Projektleiter, Trainer)?
\item
  Prüfen der Vollständigkeit mit hypothetischen Fragen, z.B. ``Was
  könnte in den ersten 3 Monaten nach ihrem Ausscheiden gegen die Wand
  laufen?'', ``Was wäre das Schlimmste, was ihrem Nachfolger passieren
  könnte? Welche Aufgaben finden sich noch in den letzten/nächsten
  Monaten in Ihrem Kalender? Welche Aufgaben machen sie regelmäßig z.B.
  täglich, wöchentlich, monatlich, quartalsweise oder jährlich?
\item
  Wie hoch ist aus Ihrer Sicht die Priorität jeder einzelnen Aufgabe
  \textbf{Hinweis:} 1 = muss, 2 = sollte, 3 = sollte erledigt werden
  \textbf{Hinweis:} sinnvolle Ebene für die Priorisierung finden, z.B.
  Ebene unterhalb der Rollen, Prioritäten nach oben aggregieren (i.d.R.
  durch Schnittbildung)
\end{enumerate}

\textbf{Abschnitt 3: Wissensgebiete}, die für die Ausführung der
Aufgaben benötigt werden, in der persönlichen Wissenslandkarte im
Bereich „Wissensgebiete`` erheben. In einem normalen Expert Debriefing
nimmt das ca. 20\% der Zeit ein. Mit diesem Teil tun sich Experten meist
am schwersten, da man so nicht über seine Arbeit nachdenkt.

\begin{enumerate}
\def\labelenumi{\arabic{enumi}.}
\tightlist
\item
  Abgeleitet aus den Aufgaben: Welche Wissensgebiete sind für die
  Ausführung Ihrer Aufgaben wichtig? \textbf{Hinweis:} Wissensgebiete in
  Lemma-Form (Grundwort, Singular) formulieren. Als Hilfe kann man dem
  Experten sagen, die Formulierung soll sein, wie bei einem
  HochschulLehrstuhl (z.B. aus Lehrstuhl für Werkstoffwissenschaften
  wird das Wissensgebiet Werkstoffwissenschaften). \textbf{Hinweis:}
  Strukturierungstiefe max. 3 Ebenen, 5±2 Einträge pro Ebene.
  \textbf{Hinweis:} wichtige Ansprechpartner, Dokumente, Tipps \& Tricks
  können gleich mit erhoben werden.
\item
  In welchen Wissensgebieten sind Sie Experte? Zu welchen
  Wissensgebieten haben Sie viel Erfahrung? Welche Fachzeitschriften,
  Newsletter, Blogs etc. lesen sie regelmäßgig? In welchen
  Fachgesellschaften sind sie Mitglied? Welche Fachkonferenzen und
  Veranstaltungen besuchen sie regelmäßig
\item
  Zu welchen Wissensgebieten werden Ihnen von anderen Personen häufig
  Fragen gestellt?
\item
  Wissensgebiete strukturieren. \textbf{Hinweis:} analog zum Aufbau von
  Wissenslandkarten gilt als Daumenregel: max. 3 Ebenen, 5±2 Einträge je
  Ebene.
\item
  Wie hoch ist aus Ihrer Sicht die Priorität jedes einzelnen
  Wissensgebiets \textbf{Hinweis:} 1=sehr wichtig, 2=wichtig, 3=weniger
  wichtig
\item
  {[}optional{]} Wissensquellen (Personen, Daten) zu den Zweigen
  eintragen.
\end{enumerate}

\textbf{Nach dem Termin:}

\begin{enumerate}
\def\labelenumi{\arabic{enumi}.}
\tightlist
\item
  Persönliche Wissenslandkarte überarbeiten (Rechtschreibung
  korrigieren, Sortierung der Zweige anpassen: nach Priorität, dann nach
  Alphabet, Farbcodes anpassen, Struktur bereinigen)
\item
  {[}optional{]} Inhalte aus Arbeitshistorie, Aufgaben und
  Wissensgebiete in den Memex (z.B. OneNote, Wiki) übertragen, um
  Grundstruktur für die Dokumentation zu schaffen.
\item
  Persönliche Wissenslandkarte als PDF exportieren, wenn die verwendete
  Mindmap-Software nicht bei allen Beteiligten verfügbar ist
\item
  Persönliche Wissenslandkarte (und ggf. Memex) an Experten schicken
  (CC-Liste aus dem Vorgespräch verwenden) und um Korrektur/Ergänzung
  bitten
\end{enumerate}

\textbf{Ressourcen und Hilfsmittel:}

\begin{itemize}
\tightlist
\item
  Exkurs Fragetechniken im Expert Debriefing im Anhang des Leitfadens
\item
  Nückles, Gurlitt, Papst: \href{https://amzn.to/2HahgC8}{Mind Maps and
  Concept Maps. Visualisieren, Organisieren, Kommunizieren}
\item
  \href{https://www.freeplane.org}{Freeplane}: kostenlose Software zur
  Erstellung von Mind Maps,
  \href{http://freemind.sourceforge.net/}{FreeMind} ist auch beliebt,
  wird aber seit 2014 nicht mehr weiterentwickelt
\item
  \href{https://markmap.js.org/}{Markmap}: Mindmapping auf Basis von
  Markdown-Dateien
\item
  \href{https://xmind.app/}{XMind}: Freemium Software zur Erstellung von
  Mind Maps, kostenlose Version reicht für Expert Debriefings
\item
  \href{http://www.mindjet.de}{MindManager}: kommerzielle Software zur
  Erstellung von Mind Maps
\end{itemize}

\textbf{Tipps und Tricks:}

\begin{itemize}
\tightlist
\item
  Die Aufgaben sollten aus Gründen der Übersichtlichkeit in einer
  flachen und nicht in einer hierarchischen Liste dokumentiert werden.
\item
  Häufig fallen dem Experten in den einzelnen Bereichen schon Maßnahmen
  ein, z.B. „da muss ich noch einen Projektbericht erstellen``. Die
  Maßnahmen entsprechend der Legende aufnehmen, so dass sie für sich
  aussagekräftig sind, z.B. „Bericht für Projekt XY fertigstellen``. Mit
  dem Mind Manager können über den Power Filter die Maßnahmen dann
  selektiert werden und per Copy \& Paste in den Maßnahmen-Plan
  überführt werden.
\item
  Frühzeitig nach Logins und Zugängen zu IT-Systemen fragen (z.B. durch
  einen Zweig ``IT-Systeme'' in der Wissenslandkarte), um rechtzeitig
  Zugänge für Nachfolger zu beantragen.
\end{itemize}

\subsubsection{Maßnahmen ableiten}\label{mauxdfnahmen-ableiten}

Die Ableitung des \textbf{Maßnahmen-Plans} dient dazu, geeignete
Maßnahmen zur Wissensbewahrung zu identifizieren, diese sowohl durch
Experten als auch durch den Nachfolger priorisieren zu lassen und
anschließend alle Maßnahmen zu terminieren. Damit wird ersichtlich,
welche Maßnahmen ggf. nicht mehr in die zur Verfügung stehende Zeit
passen.

\begin{figure}
\centering
\pandocbounded{\includegraphics[keepaspectratio,alt={Beispiel Maßnahmen-Plan in Excel}]{images/Massnahmen-Plan.png}}
\caption{Beispiel Maßnahmen-Plan in Excel}
\end{figure}

\textbf{Vorgehensweise (ca. 3 Stunden):}

\begin{enumerate}
\def\labelenumi{\arabic{enumi}.}
\tightlist
\item
  Übersicht Werkzeugkasten Expert Debriefing mit den einzelnen
  Werkzeugen erklären, damit alle Wissen, welche Möglichkeiten zur
  Verfügung stehen (Methoden wie Screencasts sind oft unbekannt).
\item
  Übernahme der bereits identifizierten Maßnahmen aus der persönlichen
  Wissenslandkarte in der Reihenfolge Prio 1, 2, 3 Themen, um den
  Aufwand für die Maßnahmen mit der zur Verfügung stehenden Zeit
  abgleichen zu können.
\item
  Bereiche „Arbeitshistorie``, „Aufgaben`` und „Wissensgebiete`` in der
  persönlichen Wissenslandkarte durchgehen und überlegen, ob sich daraus
  notwendige Maßnahmen ergeben. Diese Maßnahmen ebenfalls in den
  Maßnahmen-Plan eintragen.
\item
  Je Maßnahme die (geschätzte) Dauer und die Beteiligten in den
  Maßnahmenplan eintragen
\item
  Maßnahmen im Maßnahmen-Plan durch den Experten und Nachfolger
  priorisieren lassen. Den Maßnahmen-Plan nach Priorisierung sortieren.
\item
  Prüfen, ob der ermittelte Aufwand durch den freigegebenen Aufwand aus
  dem Vorgespräch gedeckt ist (ggf. abgestuft nach Prioritäten).
\item
  Nach dem Feedbackgespräch mit dem Vorgesetzten werden alle Maßnahmen
  endgültig terminiert. Eingetragen wird das Datum, zu dem die Maßnahme
  fertiggestellt sein soll. Den Maßnahmen-Plan nach Terminierung
  sortieren. \textbf{Tipp:} wenn die Zeit des Experten knapp ist,
  sollten schon im Vorgespräch Termine im Kalender blockiert werden.
\item
  Empfehlenswert: Regeltermin vereinbaren, um den Status der
  Maßnahmenumsetzung durchzusprechen (z.B. 30 Minute alle zwei Wochen).
\end{enumerate}

\subsubsection{Feedback einholen}\label{feedback-einholen}

Das \textbf{Einholen des Feedbacks} dient dazu, einen möglichst
objektiven Überblick über die notwendigen Maßnahmen zur Wissensbewahrung
zu erhalten und dem Auftraggeber die Möglichkeit zu geben, in den
Maßnahmen-Plan korrigierend einzugreifen.

\textbf{Vorgehensweise:}

\begin{enumerate}
\def\labelenumi{\arabic{enumi}.}
\tightlist
\item
  Persönliche Wissenslandkarte und Maßnahmen-Plan an den Vorgesetzten
  mit der Bitte um Priorisierung und Ergänzung schicken.
\item
  {[}optional{]} Persönliche Wissenslandkarte und Maßnahmen-Plan an im
  Vorgespräch festgelegte Interessensgruppen mit der Bitte um
  Priorisierung und Ergänzung schicken.
\item
  Feedback auswerten: Bei größeren Diskrepanzen zum ursprünglichen
  Maßnahmen-Plan Klärungsgespräch ansetzen. Bei kleineren Diskrepanzen
  Feedback an Maßnahmen begleiten weiterleiten.
\end{enumerate}

\textbf{Tipps und Tricks:}

\begin{itemize}
\tightlist
\item
  Ist ein Klärungsgespräch notwendig, ist eine neutrale Moderation
  zielführend.
\item
  In letzter Konsequenz zählt die Priorisierung des Auftraggebers, aber
  Vorsicht: die Entscheidung sollte sich nicht auf die Motivation des
  Experten auswirken, sein Wissen zu teilen.
\end{itemize}

\subsubsection{Maßnahmen begleiten}\label{mauxdfnahmen-begleiten}

Die Durchführung der im Maßnahmen-Plan festgelegten Maßnahmen ist der
Kern des Expert Debriefings. Ziel ist, dass Experte und Nachfolger
möglichst viele Maßnahmen in Eigenregie und in ihren Arbeitsalltag
integriert durchführt. Der Moderator hat hier zwei konkrete Rollen: 1.
Projektleiter: er wacht darüber, dass die im Maßnahmen-Plan festgelegten
Maßnahmen durchgeführt werden (ohne zu ``leitermäßig'' rüberzukommen).
2. Unterstützer: werden komplexeren Maßnahmen der Wissensbewahrung
durchgeführt (z.B. Erklärvideo, Podcast, Screencast), unterstützt der
Moderator mit Methodenwissen und Equipment.

\textbf{Vorgehensweise:}

\begin{enumerate}
\def\labelenumi{\arabic{enumi}.}
\tightlist
\item
  Im Regeltermin prüfen, ob die im Maßnahmen-Plan definierten Maßnahmen
  durchgeführt werden bzw. fest eingeplant sind. Bei Bedarf eingreifen.
\item
  Konkrete Maßnahmen, die im Maßnahmen-Plan mit „Unterstützung Moderator
  notwendig`` gekennzeichnet sind, unterstützen.
\item
  Wenn im Vorgespräch so festgelegt, wird als letzte Maßnahme die
  Wissenslandkarte des Experten an den Nachfolger übergeben
  (Signalwirkung: ``Jetzt bist du verantwortlich''). Dafür Experte um
  Freigabe bitten (bei Bedarf Inhalte löschen), persönliche
  Wissenslandkarte kopieren, Name des Nachfolgers eintragen,
  Wissenslandkarte an Nachfolger übergeben.
\end{enumerate}

\textbf{Tipps und Tricks:}

\begin{itemize}
\tightlist
\item
  Der Maßnahmen-Plan ist das Projektsteuerungsdokument für das Expert
  Debriefing. Werden Maßnahmen im Maßnahmen-Plan verschoben, sind die
  Gründe zu erfragen. Hindernisse können ggf. ausgeräumt werden (kein
  Raum, um ungestört Maßnahmen umzusetzen = Raum beschaffen). Nur im
  absoluten Notfall sollte eine Eskalation an Vorgesetzte/ Auftraggeber
  erfolgen, da sich das stark auf die Motivation auswirken kann.
\end{itemize}

\subsubsection{Reflexion moderieren}\label{reflexion-moderieren}

Die Reflexion dient der Umsetzungskontrolle sowie der kontinuierlichen
Verbesserung der Methode Expert Debriefing. Darüber hinaus sollen
Verbesserungspotentiale in der Organisation identifiziert werden, die
den Einsatz der Methode Expert Debriefing im Idealfall langfristig
überflüssig machen können. Der Termin findet i.d.R. zwischen
Auftraggeber, Experte, Nachfolger, Moderator und bei Bedarf weiteren
Beteiligten statt.

\textbf{Vorgehensweise (ca. 2 Stunden):}

\begin{enumerate}
\def\labelenumi{\arabic{enumi}.}
\tightlist
\item
  Frage an den Nachfolger: fühlen Sie sich für Ihren Job jetzt gut
  gerüstet oder gibt es noch offene Bedarfe?
\item
  Anhand des Maßnahmen-Plans und des Projektplans reflektieren (Methode:
  After-Action-Review): Was sollte aufgrund des Maßnahmen-Plans
  passieren? Was ist wirklich passiert? Was hat funktioniert? Was hat
  nicht funktioniert? Warum? Was sollten wir das nächste Mal anders
  machen?
\item
  Anhand der Maßnahmen des Maßnahmen-Plans die Frage an alle stellen:
  welche Verbesserungspotentiale sehen wir in unserer Organisation, die
  zukünftige Expert Debriefings überflüssig machen könnten?
\end{enumerate}

\textbf{Tipps und Tricks:}

\begin{itemize}
\tightlist
\item
  Das Reflexionsgespräch sollte terminlich so gelegt werden, dass noch
  die Möglichkeit besteht, fehlende Maßnahmen noch einmal zu
  priorisieren bzw. weiteren Möglichkeiten (Verfügbarkeit des Experten,
  Delegation an andere Experten/ Wissensträger) zur Umsetzung
  festzulegen.
\end{itemize}

\subsection{Expert Debriefing Toolbox}\label{expert-debriefing-toolbox}

Zur Auswahl der Maßnahmen für die Wissensbewahrung stellt die
\textbf{Expert Debriefing Toolbox} Methoden und Werkzeuge für
Moderator*innen bereit. Komplexere Werkzeuge sind in den folgenden
Kapitel im Detail beschrieben.

Die am häufigsten verwendete Methode der Wissensbewahrung ist der
\textbf{Dialog}. Dabei besprechen der Experte und der Nachfolger ein
Thema. Zu beachten ist hierbei, dass die Nachhaltigkeit eines Dialogs
begrenzt ist. So stellt sich die Frage, wie der Nachfolger z.B. nach 6
Monaten noch weiß, was alles besprochen wurde.

Eine Erweiterung des Dialogs, ist das \textbf{moderierte
Übergabegespräch}, bei dem der Moderator ein Gespräch zwischen Experte
und Nachfolger moderieren. Ein gutes Transferergebnis wird erreicht,
wenn Themen nicht nur besprochen werden, sondern \textbf{kooperatives
Arbeiten} erfolgt, d.h. der Experte und der Nachfolger führen gemeinsam
Aufgaben aus.

Um den Nachfolger systematisch zu vernetzen kommen
\textbf{Ansprechpartnerbesuche} zum Einsatz. Dabei besuchen Experte und
Nachfolger den oder die Ansprechpartner zum persönlichen Kennenlernen
und zum Vertrauensaufbau. Ist ein Besuch nicht möglich oder zu
aufwändig, können \textbf{Übergabe-Emails oder -videokonferenzen}
gemacht werden, bei dem der Experte den Nachfolger den Ansprechpartnern
vorstellt.

Weitere Bestandteile der Expert Debriefing Toolbox:

\begin{itemize}
\tightlist
\item
  \textbf{Checkliste:} Viele Aufgaben, die ein Expert „im Schlaf``
  erledigt, sind in Stellen- oder Prozessbeschreibungen nicht
  dokumentiert und können dadurch bei der Übergabe an einen Nachfolger
  vergessen werden. Die Dokumentation einer Aufgabe in Form einer
  Checkliste hilft dem Experten, implizites Wissen über die Aufgabe zu
  dokumentieren und an den Nachfolger strukturiert zu kommunizieren.
\item
  \textbf{Dokumentenbibliothek:} Experten sammeln im Laufe Ihrer
  Arbeitshistorie meist beträchtliche Mengen an Dokumenten in
  elektronischer Form und als Papier an (mit Dokument sind hier z.B.
  auch E-Mails, Links oder Weblogs gemeint). Nur der Experte hat
  Überblick über die Ablageorte (z.B. Schreibtisch, Hängeregister, PC,
  Abteilungslaufwerk, Intranet) und darüber, welche der Dokumente
  relevant sind und welche vernichtet werden können. Um dem Nachfolger
  den Zugang zu diesen Dokumenten zu ermöglichen müssen relevante
  Dokumente gefiltert, strukturiert und in einen Arbeitskontext gestellt
  werden. Die Dokumente werden dem Nachfolger in Form von einer oder
  mehrerer Dokumentenbibliotheken (en.: document repository) übergeben.
  Nicht mehr benötigte Dokumente werden gelöscht oder archiviert.
\item
  \textbf{Lessons Learned:} Fehler sind die wichtigste Quelle für das
  Lernen, sie sollten in einer Organisation aber idealerweise nur einmal
  gemacht werden. Da nicht jeder Mitarbeiter an jedem fehlerhaften
  Ereignis beteiligt ist (Primärerfahrung), werden diese Erfahrungen in
  Form von \href{https://de.wikipedia.org/wiki/Lessons_Learned}{Lessons
  Learned} oft auch
  \href{https://wissensmanagement.open-academy.com/category/methoden/wissen-dokumentieren/mikroartikel/}{Mikroartikel}
  genannt dokumentiert und können so als Sekundärerfahrung bewahrt und
  (ver-)teilt werden.
\item
  \textbf{Erklärvideo:} Bilder sagen mehr als 1.000 Worte. Kurze
  Videoaufnahmen von Versuchsaufbauten oder Sequenzen aus einer Schulung
  kombinieren Bilder mit Worten. Mit einem
  \href{https://de.wikipedia.org/wiki/Erkl\%C3\%A4rvideo}{Erklärvideo}
  können auch komplexe Sachverhalte dem Nachfolger vermittelt werden.
\item
  \textbf{FAQ (Frequently Asked Questions):} Experten sind in der
  Organisation und/oder außerhalb bekannt und beliebter Ansprechpartner
  für Fragen zu ihren Wissensgebieten. Auf die gestellten Fragen weiß
  ein Nachfolger meist noch keine Antwort. Um dem Nachfolger ein
  Hilfsmittel an die Hand zu geben, um zumindest auf die häufig
  gestellten Fragen eine Antwort geben zu können, wird gemeinsam mit dem
  Experten eine sog.
  \href{https://de.wikipedia.org/wiki/Frequently_Asked_Questions}{FAQ}
  (Frequently Asked Questions), ein Dokument mit häufig gestellten
  Fragen und zugehörigen Antworten, erstellt.
\item
  \textbf{Memex:} im Verlauf des Expert Debriefings braucht es meist
  einen Ort, an dem der Experte Wissen in Form von Checklisten,
  Ansprechpartnerlisten, Lessons Learned etc. dokumentieren kann. Dafür
  eignet sich die Idee des
  \href{https://de.wikipedia.org/wiki/Memex}{Memex} als erweitertes
  Gedächtnis. Als Memex kann z.B. ein OneNote-Notizbuch oder ein
  persönliches Wiki zum Einsatz kommen.
\item
  \textbf{Podcast:} Wissen ist immer in einen Kontext eingebettet.
  Dieser Kontext ist in einer erzählten Geschichte leichter zu
  transportieren, als in geschriebenen Text. Darüber hinaus ist es
  einfacher, etwas zu erzählen als etwas aufzuschreiben (Zitat Prusak:
  „Wir wissen immer mehr, als wir sagen können und wir sagen immer mehr,
  als wir aufschreiben können``). Vor diesem Hintergrund dient ein
  \href{https://de.wikipedia.org/wiki/Podcast}{Podcast} dazu, einen
  Zusammenhang in Audioform darzustellen und einem Nachfolger zu
  vermitteln.
\item
  \textbf{Soziales Netzwerk Diagramm:} Ein
  \href{https://de.wikipedia.org/wiki/Soziales_Netzwerk_(Soziologie)}{soziales
  Netzwerk} Diagramm zeigt Personen (Knoten des Diagramms) und
  Beziehungen zwischen Personen (Kanten des Diagramms). Das Diagramm
  zeigt somit das Beziehungsgeflecht des Experten. Das Diagramm kann
  durch eine kommentierte Ansprechpartnerliste ergänzt werden.
\item
  \textbf{Voice over PowerPoint:} Oftmals ist wichtiges implizites
  Wissen „zwischen den Spiegelstrichen`` in PowerPoint-Präsentationen
  versteckt. Bei einer Voice over PowerPoint wird eine PowerPoint-Datei
  mit einer Audiospur (ggf. auch Video) angereichert und somit leichter
  verständlich da der Text „zwischen den Zeilen`` mit aufgenommen wird.
\item
  \textbf{Screencast:} die Bedienung von Software kann man auch als
  Textdokument mit Screenshots erklären. Viel einfacher ist es jedoch,
  die Software am eigenen Rechner zu bedienen und die Interaktion
  zusammen mit den Erklärungen als Audiospur in einem
  \href{https://de.wikipedia.org/wiki/Screencast}{Screencast}
  aufzuzeichnen.
\end{itemize}

\subsubsection{Checkliste}\label{checkliste}

Viele Aufgaben, die ein Expert „im Schlaf`` erledigt, sind in Stellen-
oder Prozessbeschreibungen nicht dokumentiert und können dadurch bei der
Übergabe an einen Nachfolger vergessen werden. Die Dokumentation einer
Aufgabe in Form einer \textbf{Checkliste} hilft dem Experten, implizites
(unbewusstes) Wissen über die Aufgabe zu explizieren und an den
Nachfolger strukturiert zu kommunizieren.

\textbf{Vorgehensweise:}

\begin{enumerate}
\def\labelenumi{\arabic{enumi}.}
\tightlist
\item
  Relevante Aufgaben festlegen, für die eine Checkliste erstellt werden
  soll (z.B. alle Prio-1-Aufgaben, alle Aufgaben zu Rolle xy, Aufgaben
  x, y und z).
\item
  Vorlage für die Checklisten erstellen (z.B. OneNote, Word, Wiki, Notiz
  in der Wissenslandkarte)
\item
  Checklisten auf Basis der Vorlage erstellen. \textbf{Hinweis:} bei
  Bedarf kann zu jeder Aufgabe die Zuordnung zu einem
  (Geschäfts-)Prozess angegeben werden.
\item
  Festlegen, zu welchen Aufgaben kooperatives Arbeiten („learning by
  doing``) notwendig ist. Falls notwendig im Maßnahmen-Plan ergänzen.
  \textbf{Hinweis:} für das kooperative Arbeiten hat sich folgender
  Ablauf bewährt: Der Experte macht die Aufgabe vor, der Nachfolger
  beobachtet und kann Fragen stellen („Guided Observation``). Der
  Nachfolger macht die Aufgabe in einer nicht „scharf geschalteten``
  Umgebung nach, der Experte kann korrigieren. Der Nachfolger führt die
  Aufgabe eigenständig aus, der Experte agiert als Coach.
\end{enumerate}

\textbf{Ressourcen und Hilfsmittel:}

\begin{itemize}
\tightlist
\item
  Webseiten wie \href{https://wikihow.com}{wikiHow} oder
  \href{https://www.ehow.com}{eHow} bieten schöne Beispiele für
  Checklisten und Tutorials.
\end{itemize}

\subsubsection{Dokumentenbibliothek}\label{dokumentenbibliothek}

Experten sammeln im Laufe Ihrer Arbeitshistorie meist beträchtliche
Mengen an Dokumenten in elektronischer Form und als Papier an (mit
Dokument sind hier z.B. auch E-Mails, Links oder Weblogs gemeint). Nur
der Experte hat Überblick über die Ablageorte (z.B. Schreibtisch,
Hängeregister, PC, Abteilungslaufwerk, Intranet) und darüber, welche der
Dokumente relevant sind und welche vernichtet werden können (oftmals
mehr als 90\%).

Um dem Nachfolger den Zugang zu diesen Dokumenten zu ermöglichen müssen
relevante Dokumente gefiltert, strukturiert und in einen Arbeitskontext
gestellt werden. Die Dokumente werden dem Nachfolger in Form von einer
oder mehrerer \textbf{Dokumentenbibliotheken} (en.: document repository)
übergeben. Nicht mehr benötigte Dokumente werden gelöscht oder
archiviert.

\textbf{Vorgehensweise:}

\begin{enumerate}
\def\labelenumi{\arabic{enumi}.}
\tightlist
\item
  Relevante Dokumentquellen definieren, z.B. Schreibtisch,
  Ordnungssysteme im Büro (z.B. Regal, Aktenschrank, Hängeregister),
  Ordnungssysteme außerhalb des Büros (z.B. Archiv), Persönlicher PC,
  Persönliches Laufwerk, Abteilungslaufwerk, Projektlaufwerke, Intranet
  und andere Informationssystem
\item
  Ziel(e) für die Dokumentenbibliothek(en) definieren, z.B. Laufwerk,
  USB-Stick.
\item
  Dokumentquellen sichten und Ablagestruktur für die
  Dokumentenbibliothek(en) definieren.
\item
  Dokumente durch den Experten (oder mit dem Experten) sichten,
  aussortieren und in die Dokumentenbibliothek(en) überführen (lassen).
\item
  {[}optional{]} Dokumentenbibliothek(en) mit der persönlichen
  Wissenslandkarte bzw. dem persönlichen Wiki verlinken.
\item
  Dokumentenbibliothek(en) in einem moderierten Übergabegespräch mit
  Experte und Nachfolger durchsprechen (ggf. in Screencast).
\end{enumerate}

\textbf{Ressourcen und Hilfsmittel:}

\begin{itemize}
\tightlist
\item
  Margit Gätjens-Reuter: \href{https://amzn.to/3jZWIuB}{Ablage.
  Information optimal organisieren}.
\item
  Steinbrecher, W., Müll-Schnurr, M.:
  \href{https://amzn.to/3lQNQro}{Prozessorientierte Ablage}.
\end{itemize}

\subsubsection{Lessons Learned}\label{lessons-learned}

Fehler sind die wichtigste Quelle für das Lernen, sie sollten in einer
Organisation aber idealerweise nur einmal gemacht werden. Da nicht jeder
Mitarbeiter an jedem Ereignis, bei dem Fehler gemacht wurden, beteiligt
ist (Primärerfahrung), werden diese Erfahrungen in Form von Lessons
Learned als Mikroartikeln dokumentiert und können so als
Sekundärerfahrung bewahrt und (ver-)teilt werden.

\textbf{Vorgehensweise:}

\begin{enumerate}
\def\labelenumi{\arabic{enumi}.}
\tightlist
\item
  Relevante Ereignisse definieren, zu denen Erfahrungen dokumentiert
  werden sollen (z.B. Projekte, Restrukturierungen, Stellenwechsel,
  Vorgänge).
\item
  Medium für die Lessons-Learned-Liste und die Dokumentation der
  Mikroartikel festlegen, z.B. OneNote, Word, Wiki), bestehende
  Lessons-Learned-System (z.B. Datenbanken).
\item
  Vorlage für die Mikroartikel definieren, z.B. wie Mikroartikel mit
  Thema, Geschichte, Einsicht, Folgerung, Anschlussfragen.
\item
  Themen in Lessons-Learned-Liste eintragen
\item
  Je Lessons Learned einen Mikroartikel erstellen
\item
  Lessons Learned in einem moderierten Übergabegespräch mit Experte und
  Nachfolger durchsprechen.
\end{enumerate}

\textbf{Ressourcen und Hilfsmittel:}

\begin{itemize}
\tightlist
\item
  Willke: \href{https://amzn.to/3lNdg9t}{Systemisches
  Wissensmanagement}.
\end{itemize}

\subsubsection{Erklärvideo}\label{erkluxe4rvideo}

Bilder sagen mehr als 1.000 Worte. Kurze Videoaufnahmen von
Versuchsaufbauten oder Sequenzen aus einer Schulung kombinieren Bilder
mit Worten. Damit können auch komplexe Sachverhalte dem Nachfolger
vermittelt werden.

\textbf{Vorgehensweise:}

\begin{enumerate}
\def\labelenumi{\arabic{enumi}.}
\tightlist
\item
  Relevante Themen identifizieren, zu denen Erklärvideos erstellt werden
  sollen.
\item
  Produktionsumgebung definieren, z.B. Kamera (Webcam, Smartphone,
  Digitalkamera, Camcorder) und Software (Logitech Capture, Davinci
  Resolve, KDEnlive, Windows Video Editor, OneNote)
\item
  Zielgruppe für das Erklärvideo festlegen (Standard: nur Nachfolger)
  \textbf{Hinweis:} wichtig für die Inhalte, da schneiden eines Videos
  hinterher aufwändig!
\item
  \href{https://de.wikipedia.org/wiki/Storyboard}{Storyboard} festlegen
  (was wird wie aufgenommen)
\item
  Video aufzeichnen.
\item
  Video nachbearbeiten (z.B. Anfang und Ende schneiden, kritische
  Stellen schneiden).
\item
  Video an Nachfolger übergeben.
\end{enumerate}

\textbf{Ressourcen und Hilfsmittel:}

\begin{itemize}
\tightlist
\item
  \href{http://l3t.tugraz.at/index.php/LehrbuchEbner10/article/view/38}{Interaktive,
  multimediale Materialien - Gestaltung von Materialien zum Lernen und
  Lehren} im Lehrbuch für Lernen und Lehren mit Technologien.
\end{itemize}

\subsubsection{FAQ (Frequently Asked
Questions)}\label{faq-frequently-asked-questions}

Experten sind in der Organisation (und meist auch außerhalb) bekannt und
beliebter Ansprechpartner für Fragen zu ihren Wissensgebieten, auf die
ein Nachfolger meist noch keine Antwort weiß. Um dem Nachfolger ein
Hilfsmittel an die Hand zu geben, um zumindest auf die am häufigsten
gestellten Fragen eine Antwort geben zu können, wird gemeinsam mit dem
Experten eine sog. FAQ (Frequently Asked Questions), ein Dokument mit
häufig gestellten Fragen und deren Antworten, erstellt.

\textbf{Vorgehensweise:}

\begin{enumerate}
\def\labelenumi{\arabic{enumi}.}
\tightlist
\item
  Relevante Wissensgebiete, zu denen FAQs erstellt werden sollen,
  definieren.
\item
  Medium für die Dokumentation festlegen, z.B. OneNote, Word, Wiki
\item
  Je Wissensgebiet häufig gestellte Fragen formulieren und festhalten
  \textbf{Hinweis:} wenn es sich um mehr als 10 Fragen handelt, kann es
  sinnvoll sein, die Fragen in mehrere Fragengruppen zu gruppieren und
  jeder Fragengruppe einen Namen zu geben.
\item
  Antworten zu häufig gestellten Fragen formulieren und festhalten
\item
  FAQs in einem moderierten Übergabegespräch mit Experte und Nachfolger
  durchsprechen.
\end{enumerate}

\textbf{Ressourcen und Hilfsmittel:}

\begin{itemize}
\tightlist
\item
  Gute Beispiele zu FAQs gibt es im \href{http://www.faqs.org}{Internet
  FAQ Archiv}
\end{itemize}

\subsubsection{Memex}\label{memex}

Das Gedächtnis eines jeden Menschen ist begrenzt. Deswegen wird im Zuge
eines Expert Debriefing ein Memex zur Wissensdokumentation eingesetzt,
in das der Experte beliebige Eintragungen machen kann. Die
Wahrscheinlichkeit, dass wichtige Dinge vergessen werden, sinkt dadurch
stark. Der Memex kann z.B. in Form eines OneNote-Notizbuchs oder als
Wiki angelegt werden.

\textbf{Vorgehensweise:}

\begin{enumerate}
\def\labelenumi{\arabic{enumi}.}
\tightlist
\item
  Tool für den Memex festlegen und gemeinsam mit dem Experten einrichten
  (z.B. OneNote, persönliches Wiki)
\item
  Kurzeinweisung für Experten geben, damit dieser den Memex bedienen
  kann.
\item
  Inhalte im Memex erstellen (kontinuierlich während des Expert
  Debriefings).
\item
  Memex an Nachfolger übergeben und in einem Übergabegespräch mit
  Experte und Nachfolger durchsprechen.
\end{enumerate}

\textbf{Ressourcen und Hilfsmittel:}

\begin{itemize}
\tightlist
\item
  Wikipedia-Artikel \href{https://de.wikipedia.org/wiki/Memex}{Memex}
\end{itemize}

\subsubsection{Podcast}\label{podcast}

Wissen ist immer in einen Kontext eingebettet. Dieser Kontext ist in
einer erzählten Geschichte leichter zu transportieren, als in
geschriebenen Text. Darüber hinaus ist es einfacher, etwas zu erzählen
als etwas aufzuschreiben.

\begin{quote}
Wir wissen immer mehr, als wir sagen können und wir sagen immer mehr,
als wir aufschreiben können - Larry Prusak
\end{quote}

Vor diesem Hintergrund dient ein Podcast dazu, einen Zusammenhang in
Audioform darzustellen und einem Nachfolger zu vermitteln. Podcasts
werden als Werkzeug immer dann eingesetzt, wenn der Nachfolger nicht an
den Maßnahmen teilnehmen kann, z.B. als Podcast mit Durchsprache der
Aufgaben der persönlichen Wissenslandkarte.

\textbf{Vorgehensweise:}

\begin{enumerate}
\def\labelenumi{\arabic{enumi}.}
\tightlist
\item
  Relevante Themen identifizieren, zu denen Podcasts erstellt werden
  sollen.
\item
  Produktionsumgebung definieren, z.B. Mikrofon, Audio-Rekorder,
  Audio-Editor
\item
  Zielgruppe für den Podcast festlegen (Standard: nur Nachfolger)
  \textbf{Hinweis:} wichtig für die Inhalte, da schneiden eines Podcasts
  hinterher aufwändig!
\item
  Podcast aufzeichnen.
\item
  Podcast nachbearbeiten (z.B. Anfang und Ende schneiden, kritische
  Stellen schneiden).
\item
  Podcast an Nachfolger übergeben. \textbf{Hinweis:} sicherstellen, dass
  Nachfolger über die Infrastruktur verfügt, um Podcasts anhören zu
  können (z.B. Audioplayer, Kopfhörer).
\end{enumerate}

\textbf{Ressourcen und Hilfsmittel:}

\begin{itemize}
\tightlist
\item
  \href{http://audacity.sourceforge.net}{Audacity} - freier Audioeditor
  und --rekorder.
\item
  Sprachmemo App auf dem Smartphone (z.B. iPhone)
\item
  \href{https://cogneon.github.io/lernos-podcasting/de/}{lernOS
  Podcasting Leitfaden} mit vielen Tipps und Tricks rund um die
  Audioproduktion
\end{itemize}

\subsubsection{Soziales Netzwerk
Diagramm}\label{soziales-netzwerk-diagramm}

Ein soziales Netzwerk Diagramm (auch Beziehungslandkarte) zeigt Personen
(Knoten des Diagramms) und Beziehungen zwischen Personen (Kanten des
Diagramms). Das Diagramm zeigt somit das Beziehungsgeflecht des
Experten. Das Diagramm kann durch eine kommentierte Ansprechpartnerliste
mit Hinweisen zu Personen und Institutionen ergänzt werden.

\begin{figure}
\centering
\pandocbounded{\includegraphics[keepaspectratio,alt={Beispiel Soziales Netzwerk Diagramm, mit yED erstellt}]{images/Soziales-Netzwerk-Diagramm.png}}
\caption{Beispiel Soziales Netzwerk Diagramm, mit yED erstellt}
\end{figure}

\textbf{Vorgehensweise:}

\begin{enumerate}
\def\labelenumi{\arabic{enumi}.}
\tightlist
\item
  Relevante Ansprechpartner in Ansprechpartnerliste eintragen.
\item
  Welche Personen sind für die Ausführung der Aufgaben wichtig?
  \textbf{Hinweis:} Personen können z.B. Kunden, Lieferanten, Partner,
  Interessensgruppen (Stakeholder), Wettbewerber, Verbände sein.
\item
  Welche Personen kontaktieren Sie häufig?
\item
  Welche Personen sind für Sie bei der Lösung von Problemen wichtig?
\item
  Zu welchen Personen halten Sie regelmäßig Kontakt, um informiert zu
  sein?
\item
  Für wen sind Sie wichtig? Wer kontaktiert Sie häufig?
\item
  Bei Bedarf: auch Organisationen, Projekte, Städte, Länder, Regionen
  etc. im Diagramm eintragen.
\item
  Beziehungen zwischen Ansprechpartnern (ggf. über Organisationen) in
  sozialem Netzwerk Diagramm darstellen.
\end{enumerate}

\textbf{Ressourcen und Hilfsmittel:}

\begin{itemize}
\tightlist
\item
  \href{http://www.yworks.com}{yED} - Grafikeditor, zur Darstellung
  eines soziales Netzwerk Diagramms. Wichtige Funktion ist das
  automatische layouten des Netzwerks, da das sonst sehr viel manuellen
  Aufwand erzeugt.
\end{itemize}

\subsubsection{Voice over PowerPoint}\label{voice-over-powerpoint}

Oftmals ist wichtiges implizites Wissen „zwischen den Spiegelstrichen``
in PowerPoint-Präsentationen versteckt. Bei einer Voice over PowerPoint
wird eine PowerPoint-Datei mit einer Audiospur (ggf. auch Video)
angereichert und somit leichter verständlich da der gesprochene Text mit
aufgenommen wird. Der Aufwand für den Experten ist deutlich geringer,
als bei der Dokumentation in den Textnotizen von Powerpoint. Die
Audioaufzeichnung kann hinterher je Folie ausgetauscht werden, so dass
multimediale PowerPoints leicht an sich verändernden Inhalt angepasst
werden können.

\textbf{Vorgehensweise:}

\begin{enumerate}
\def\labelenumi{\arabic{enumi}.}
\tightlist
\item
  Mikrofon an das Notebook anschließen (z.B. Videokonferenz-Headset,
  oder Freisprecheinrichtung, wenn mehrere Personen sprechen)
\item
  PowerPoint-Datei öffnen
\item
  Aufzeichnung starten: Bildschirmpräsentation -\textgreater{}
  Bildschirmpräsentation aufzeichnen (dort können auch Mikrofon und
  Kamera ausgewählt werden)
\item
  Aufzeichnung starten
\item
  Text einsprechen (1-2 Sek. Pause zum Folienwechsel). Um eine hohe
  Flexibilität zu gewährleisten, soll der Text pro Seite bzw. logischer
  Einheit gesprochen werden. Dadurch können später einzelne Seiten und
  deren Texte ausgetauscht werden, ohne die Gesamtaufnahme wiederholen
  zu müssen. \textbf{Hinweis:} zwischen den Folien etwas Pause lassen,
  das erleichtert die Nachbearbeitung bei Fehlern oder Aktualisierungen.
\item
  Aufzeichnung stoppen und PowerPoint-Datei speichern
\item
  Probehören, bei Bedarf einzelne Folien neu vertonen
\end{enumerate}

\textbf{Ressourcen und Hilfsmittel:}

\begin{itemize}
\tightlist
\item
  Hilfe-Seite
  \href{https://support.microsoft.com/de-de/office/aufzeichnen-einer-bildschirmpr\%C3\%A4sentation-mit-kommentaren-und-folienanzeigedauern-0b9502c6-5f6c-40ae-b1e7-e47d8741161c}{Aufzeichnen
  einer Bildschirmpräsentation mit Kommentaren und Folienanzeigedauern}
  von Microsoft
\end{itemize}

\subsubsection{Screencast}\label{screencast}

die Bedienung von Software kann man auch als Textdokument mit
Screenshots erklären. Viel einfacher ist es jedoch, die Software am
eigenen Rechner zu bedienen und die Interaktion zusammen mit den
Erklärungen als Audiospur aufzuzeichnen.

\paragraph{Vorgehensweise}\label{vorgehensweise}

\begin{enumerate}
\def\labelenumi{\arabic{enumi}.}
\tightlist
\item
  Tool für die Screencast-Erstellung auswählen, z.B. PowerPoint,
  OneNote, , Snipping Tool, Snagit, Camtasia, Screenpresso, MS Teams,
  Zoom, OBS.
\item
  Text für die Tonspur stichpunktartig überlegen.
\item
  Desktop aufräumen und darauf achten, dass keine Fenster mit sensiblen
  Daten im Screencast zu sehen sind.
\item
  Screencast aufnehmen. Dabei auf gutes Mikrofon und ruhige Umgebung
  achten.
\item
  Screencast bei Bedarf schneiden.
\end{enumerate}

\section{Lernpfad}\label{lernpfad}

\begin{enumerate}
\def\labelenumi{\arabic{enumi}.}
\tightlist
\item
  \textbf{Teste deine Schreibgeschwindigkeit:} gehe auf
  \href{https://www.zehnfinger.com/test.php}{zehnfinger.com} und teste
  deine Schreibgeschwindigkeit und Fehlerrate. Ziel sollte eine
  Geschwindigkeit von größer 200 Anschlägen pro Minute und eine
  niedrige, einstellige Fehlerrate sein. Zur Verbesserung der
  Schreibgeschwindigkeit kannst du Tools wie
  \href{https://portableapps.com/apps/education/typefaster_portable}{TypeFaster}
  oder
  \href{https://portableapps.com/apps/education/tipp10_portable}{TIPP10}
  verwenden.
\item
  \textbf{Lerne deine MindMap-Software kennen:} wähle die
  MindMap-Software, die du im Expert Debriefing verwenden möchtest. Oft
  verwendet werden Freeplane (ggf. Freemind, wird aber nicht mehr
  weiterentwickelt), XMind und MindManager. Mache dich mit den
  Funktionen und Tastaturkürzeln vertraut, um die Software möglichst
  schnell bedienen zu können.
\item
  \textbf{Strukturiere einen Text als MindMap:} verwende den Artikel
  \href{https://www.dgfp.de/hr-wiki/Wissenstransfer_und_organisationales_Lernen_mit_Expert_Debriefing_und_Wikis.pdf}{Wissenstransfer
  und organisationales Lernen mit Expert Debriefing und Wikis} von Karin
  Hartmann und Simon Dückert, um diesen in eine strukturierte MindMap zu
  überführen.
\item
  \textbf{Strukturiere einen Podcast als MindMap:} verwende den Podcast
  \href{https://cogneon.de/2015/02/05/m2p013-14-jahre-wissensmanagement-bei-schaeffler/}{14
  Jahre Wissensmanagement bei Schaeffler} mit Paul Seren und Simon
  Dückert, um diesen in eine strukturierte MindMap zu überführen. Halt
  den Podcast dabei möglichst selten an.
\item
  \textbf{Baue deine eigene Wissenslandkarte auf:} verwende die
  Anleitung im Grundlagenkapitel dieses Leitfadens, um deine eigene
  Wissenslandkarte aufzubauen. Exportiere die Wissenslandkarte auch in
  einem Format, dass du Personen ohne die MindMap-Software weitergeben
  kannst.
\item
  \textbf{Erstelle dein eigenes Expert Debriefing Szenario:} verwende
  die Anleitung zum Vorgespräch im Grundlagenkapitel, um ein mögliches
  Szenario zu entwickeln, bei dem Wissensbewahrung in deiner Situation
  notwendig werden könnte (z.B. Unternehmenswechsel, längere
  Abwesenheit, Übergabe eines Teils der Aufgaben).
\item
  \textbf{Starte ein kleines Praxisprojekt:} suche dir eine
  Versuchsperson, mit der du ein kleines Praxisprojekt durchführen und
  darin alle Schritte des Expert Debriefing Referenzprozess ausprobieren
  kannst. Erkläre der Person den Expert Debriefing Prozess und achte
  darauf, welche Fragen gestellt werden.
\item
  \textbf{Führe das Vorgespräch und baue die Wissenslandkarte auf:}
  verwende einen ersten Termin, um darin das Vorgespräch zu führen und
  im Anschluss die Wissenslandkarte aufzubauen. Im ersten Praxisprojekt
  können die Inhalte der Wissenslandkarte exemplarisch schein, um die
  benötigte Zeit zu verkürzen.
\item
  \textbf{Leite den Maßnahmen-Plan ab:} leite im Folgetermin den
  Maßnahmen-Plan gemeinsam mit deiner Testperson ab. Schaut euch dabei
  explizit auch die Werkzeuge der Expert Debriefing Toolbox an und
  überlegt, welche Tools und Methoden in diesem Praxisprojekt sinnvoll
  sein könnten.
\item
  \textbf{Experimentiere mit der Expert Debriefing Toolbox:} probiere
  Maßnahmen aus, die in deinen Expert Debriefings in Zukunft relevant
  sein könnten. Experimentiere dabei auch mit Methoden und Tools, mit
  denen du bisher noch keine Erfahrung hast, z.B. OneNote, Screencast,
  Podcasts, Voice-over-PowerPoint oder sozialen Netzwerkdiagrammen.
\item
  \textbf{Führe das Reflexionsgespräch:} verwende ca. 30 Minuten für ein
  Reflexionsgespräch. Frage deine Testperson dabei auch nach konkretem
  Feedback zu dir als Moderator. Das Praxisprojekt ist eine gute Chance,
  sehr offenes und ehrliches Feedback zu bekommen.
\end{enumerate}

\textbf{ProTip:} in der Kategorie
\href{https://community.cogneon.de/c/topics/expert-debriefing}{Expert
Debriefing} auf der Community Plattform CONNECT kannst du dich mit
anderen Moderatoren vernetzten, Fragen stellen und Erfahrungen teilen.

\section{Anhang}\label{anhang}

\subsection{Exkurs: Fragetechniken im Expert
Debriefing}\label{exkurs-fragetechniken-im-expert-debriefing}

Der Aufbau einer persönlichen Wissenslandkarte erfolgt als Dialog
zwischen dem Moderator und dem Experten (ggf. ist auch der Nachfolger
anwesend).

Als Moderator setzen Sie Fragetechniken ein. Das Ziel der angewandten
Fragetechnik ist es, den Dialog zu beginnen bzw. zu vertiefen und den
Gesprächspartner dabei rhetorisch zu lenken. (Wichtig: Lenken und nicht
manipulieren!)

Die Fragetechniken können klassifiziert werden, z.B. nach

\begin{itemize}
\tightlist
\item
  Offene Fragen
\item
  Geschlossene Fragen
\item
  Reflektierende Fragen
\item
  Hypothetische Fragen
\end{itemize}

Im Expert Debriefing kommen immer Mischformen vor. Hinter den Beispielen
sind die Anwendungsbereiche im Prozess aufgeführt (pWLK = Persönliche
Wissenslandkarte allgemein/ SND = Soziales Netzwerkdiagramm)

\textbf{Offene Fragen:}

\begin{itemize}
\tightlist
\item
  Was haben Sie vorher alles gemacht? (Arbeitshistorie)
\item
  In welchen Rollen sind sie unterwegs? ( Aufgaben)
\item
  Was waren Ihre Aufgaben? (Aufgaben/ Arbeitshistorie)
\item
  Welche Ansprechpartner hatten Sie? (pWLK/SND)
\item
  Woran machen Sie das fest? (Aufgaben)
\item
  Wer kann dazu noch etwas sagen/ beitragen? (pWLK/SND)
\item
  Wie würden Sie das formulieren? (pWLK/SND)
\end{itemize}

\textbf{Geschlossene Fragen:}

\begin{itemize}
\tightlist
\item
  Wie lange wollen wir in der Arbeitshistorie zurückgehen?
  (Arbeitshistorie)
\item
  Wo ist das eingeflossen? (Frage nach der Dokumentation)
\item
  Ist „A`` wichtiger als „B``? -- (Priorisierung der Aufgaben)
\item
  Ist das so korrekt aufgenommen? -- (Erstellung der pWLK)
\end{itemize}

\textbf{Reflektierende Fragen:}

\begin{itemize}
\tightlist
\item
  Habe ich das richtig verstanden, dass \ldots{} (pWLK/ Podcast)
\end{itemize}

\textbf{Hypothetische Fragen:}

\begin{itemize}
\tightlist
\item
  Was ist das schlimmste, was ihrem Nachfolger passieren könnte?
\item
  Was kann in den ersten 3 (6/ 9) Monaten nach ihrem Weggang passieren?
\item
  Wenn Sie noch 3 Monate länger bleiben würden, was würden Sie dann
  machen?
\item
  Welche Ideen haben Sie, die noch nicht umgesetzt wurden?
\item
  Wenn Sie etwas ändern könnten, was wäre es?
\end{itemize}

\subsection{lernOS Leitfaden mit KI-Chatbots
nutzen}\label{lernos-leitfaden-mit-ki-chatbots-nutzen}

Zusätzlich zu den gut ``menschenlesbaren'' Formaten der lernOS Leitfäden
(z.B. PDF, E-Book) generieren wir mit der Markdown-Version auf eine
****``maschinenlesbare'' Variante des Leitfadens***. Die kann sehr gut
mit KI-Chatbots wie z.B. ChatGPT, Microsoft Copilot, Claude Gemini oder
auch lokalen KI-Modellen (z.B. gpt-oss, qwen) verwendet werden.

Dazu einfach in der Web-Version des Leitfadens unter \emph{Downloads}
die Markdown-Datei herunterladen (Datei mit der Endung .md) und in den
Chatbot der Wahl hochladen. Im folgenden einige Idee für Prompts zum
Start:

Unterstützung bei der \textbf{Einführung in das Thema} Expert
Debriefing:

\begin{quote}
ROLE: Du bist ein Experte in Expert Debriefing, einer Methode zur
Bewahrung von implizitem Wissen. Du hilfst mir, die Grundlagen der
Methode zu verstehen. Ich habe dir den Expert Debriefing Leitfaden
beigefügt, verwende deine Informationen ausschließlich daraus. Wenn
benötigte Information nicht im Leitfaden enthalten ist, sage dass du
über die gewünschte Information nicht verfügst und erfinde keine eigenen
Informationen. TASK: Teste in einem ersten Schritt in sachlichem aber
unterhaltsamen Ton, welches Wissen aus dem Kapitel ``Grundlagen'' ich
schon habe. Stelle mir die Fragen eine nach der anderen und nicht alle
auf einmal. Erstelle mir dann im zweiten Schritt einen personalisierten
Lernplan, der Schritt für Schritt meine Wissenslücken schließt. Führe im
dritten und letzten Schritt einen Multiple-Choice-Test mit mir durch, um
mein Wissen über Expert Debriefing zu prüfen.
\end{quote}

Lerncoach für das \textbf{Erlernen des Referenzprozesses} Expert
Debriefing:

\begin{quote}
ROLE: Du bist ein Experte in Expert Debriefing, einer Methode zur
Bewahrung von implizitem Wissen. Du hilfst mir, den Expert Debriefing
Referenzprozess mit allen Tipps und Hilfsmitteln im Detail zu verstehen.
Den Leitfaden zum Expert Debriefing habe ich beigefügt, verwende deine
Informationen ausschließlich daraus. Wenn benötigte Information nicht im
Leitfaden enthalten ist, sage dass du über die gewünschte Information
nicht verfügst und erfinde keine eigenen Informationen. TASK: Schritt 1:
erkläre mir den jeweiligen Prozessschritt in einer kompakten Übersicht.
Die sechs Schritte des Expert Debriefing Referenzprozesses sind
Vorgespräch führen, Wissenslandkarte aufbauen, Maßnamen ableiten,
Feedback einholen, Maßnahmen begleiten und Reflexion moderieren. Frage
mich, welche Fragen ich habe und hilf mir bei der Beantwortung so lange,
bis keine Fragen mehr offen sind. Wiederhole Schritt 1 für alle
Prozessschritte. Schritt 2: spiele folgendes Szenario durch: du bist der
Moderator und ich der Experte. Spiele mit mir im Rollenspiel ein Expert
Debriefing in ca. 15-30 Minuten durch, so dass ich die wesentlichen
Bestandteile des Prozesses erkennen kann.
\end{quote}

Hilfe bei der \textbf{Erstellung eines Gesprächsleitfadens} für den
Aufbau der Wissenslandkarte:

\begin{quote}
ROLE: Du bist ein Experte in Expert Debriefing, einer Methode zur
Bewahrung von implizitem Wissen. Du hilfst mir, einen Gesprächsleitfaden
für die Erstellung der Wissenslandkarte vorzubereiten. Den Leitfaden zum
Expert Debriefing habe ich beigefügt, verwende deine Informationen
ausschließlich daraus. Wenn benötigte Information nicht im Leitfaden
enthalten ist, sage dass du über die gewünschte Information nicht
verfügst und erfinde keine eigenen Informationen. TASK: Verwende (wenn
verfügbar) das beigefügte Protokoll des Vorgesprächs mit Informationen
wie Hauptaufgaben oder wichtigen Wissensgebieten des Experten. Erstelle
einen Gesprächsleitfaden mit Zeitangaben, damit ich nach Vorgabe des
Expert Debriefing Leitfadens eine Wissenslandkarte mit den drei
Bereichen Arbeitshistorie, Aufgaben und Wissensgebiete erstellen und
priorisieren kann. Achte bei den Fragen darauf, dass sie nicht nur
explizites Wissen (bewusstes, regelhaftes, dokumentiertes Wissen)
ansprechen, sondern auch das implizite Wissen des Experten (Erfahrungen,
Heuristiken, Haltungen, Bauchgefühl) sichtbar machen.
\end{quote}

z\#\# Änderungshistorie

{\def\LTcaptype{none} % do not increment counter
\begin{longtable}[]{@{}
  >{\raggedright\arraybackslash}p{(\linewidth - 6\tabcolsep) * \real{0.0317}}
  >{\raggedright\arraybackslash}p{(\linewidth - 6\tabcolsep) * \real{0.1267}}
  >{\raggedright\arraybackslash}p{(\linewidth - 6\tabcolsep) * \real{0.7964}}
  >{\raggedright\arraybackslash}p{(\linewidth - 6\tabcolsep) * \real{0.0452}}@{}}
\toprule\noalign{}
\begin{minipage}[b]{\linewidth}\raggedright
Version
\end{minipage} & \begin{minipage}[b]{\linewidth}\raggedright
Bearbeitet von
\end{minipage} & \begin{minipage}[b]{\linewidth}\raggedright
Beschreibung Änderung
\end{minipage} & \begin{minipage}[b]{\linewidth}\raggedright
Datum
\end{minipage} \\
\midrule\noalign{}
\endhead
\bottomrule\noalign{}
\endlastfoot
1.0 & Simon Dückert & Erstellung des Expert Debriefing Leitfadens im
Rahmen des
\href{https://cloud.cogneon.de/s/WiiRfj64CFoc3g2}{Expert-Debriefing-Einführungsprojektes
bei Schaeffler}. & 2006 \\
2.0 & Simon Dückert & Beschreibung des neuen Prozesses, Überarbeitung
des Werkzeugkastens. & 09.02.2009 \\
2.1 & Marc Nitschke, Simon Dückert & Überarbeitung des Prozesses,
Integration des Werkzeugkastens. & 23.06.2011 \\
2.2 & Marc Nitschke & Ergänzungen in 2.1 Vorgespräch führen, 2.3
Persönliche Wissenslandkarte aufbauen und 2.4 Maßnahmen ableiten. &
08.10.2012 \\
2.3 & Marc Nitschke & Einfügung der Kapitel 1.2 Erfolgsfaktoren für die
Durchführung und 2.2 Exkurs: Fragetechniken. & 21.11.2012 \\
2.4 & Simon Dückert & Überführung des Leitfadens in das Format der
lernOS Leitfäden. Keine inhaltlichen Anpassungen. & 19.11.2018 \\
3.0 & Simon Dückert & Überarbeitung des Grundlagen-Kapitels,
Aktualisierung der Toolbox und der Tool-Beschreibungen, Ergänzung der
Übungen aus der Cogneon Ausbildung zum Expert Debriefing Moderator. &
05.10.2020 \\
3.1 & Simon Dückert & Erzeugung einer Markdown-Version für KI-Tools mit
Beschreibung im Anhang. & 18.11.2025 \\
\end{longtable}
}

\end{document}
